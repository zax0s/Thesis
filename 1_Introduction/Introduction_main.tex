\chapter{Introduction}
Positron Emission Tomography (PET) has grown over a period of multiple decades from a tool used infrequently and primarily limited to research applications into a clinical imaging tool which plays a major role in many clinical practices and applications. 
In combination with multiple available radiotracers, the compounds that enable PET imaging to target and gather information of function at the molecular level, PET can provide unique biomarker information. The non-invasive nature of PET imaging, in combination with its potential for fully quantitative and reliable reproducibility of biomarker information are some very important aspects that make it desirable in delivery of precision medicine. Precision medicine is defined as an approach for disease treatment and prevention that takes into account individual variability in genes, environment, and lifestyle for each person. PET can play an vital role towards wider adoption of precision medicine due to its potential in delivery of unique biomarker information~\cite{Subramaniam2017}.

Engineering advancements over the last decades have led to the development of highly efficient PET systems and in the creation of Hybrid imaging systems, notably the PET/CT and PET/MR systems. The clinical need for multi-modality information has led to fast adoption of such hybrid systems in clinical practice, with “one-stop-shop” protocols providing mutli-modality information in single imaging sessions. Finally, hybrid systems have enabled use of imaging information in “synergy”, to provide new and superior information on underlying function and anatomy. 

Despite PET’s great success in clinical integration and applicability, the use of the modality and the main focus of many years of development have been concentrated on static imaging and qualitative or semi-quantitative measures. These have been driven mainly by applications in oncology, where these measures have so far been deemed sufficient in clinical pathways of cancer patients. At the same time, the ability of PET to provide fully quantitative measures of the biological functions under study has been reserved for research orientated applications and developments of new tracers or study of human physiology. 
But recently, the increasing focus on precision medicine has led to renewed attention to PET’s dynamic capabilities, for quantification of biologically relevant parameters by monitoring tracer kinetics and dynamic behaviour. These quantitative capabilities of PET are expected to be at the centre of future developments and clinical research for the next decade~\cite{Lammertsma2017,Meikle2021}.

\subsection*{Problemata}
The majority of the current clinical applications of PET can be found in oncology, where imaging information over the whole body is needed to detect and characterise primary and metastatic disease. 
Whole body information though is desirable for many current and potential future applications of dynamic PET, that can result in fully quantitative measures and biomarkers information over the whole body. Beyond the use of such information for precision oncology, it can be used to study kinetics of drugs distribution and function over the whole body.
This information can be used, while considering the human body as a single system, in the study of interactions and signalling between organs of the body to study complex physiology and pathology interactions.
A major limitation in acquisition of the required PET data over the whole body is the limited axial field of view (A-FOV) of the majority of current clinical systems. The majority of the PET systems provide an axial coverage between 15 and 26 cm ~\cite{Vandenberghe2020}. 
In practice whole-body coverage is achieved using multiple bed positions at different axial locations to provide the desired axial coverage~\cite{Schubert1996}, or alternatively via continuous axial bed motion (CBM) during the acquisition~\cite{Panin2014}. 
Both acquisition strategies can be extended for dynamic whole-body acquisition, by use of multiple repeated whole-body passes~\cite {Karakatsanis2011,Karakatsanis2013,Rahmim2019}.
More recently these dynamic whole-body protocols have been integrated within some commercial PET systems~\cite{Hu2020}. 
These acquisition protocols pose challenges that arise from the result temporal gaps in the acquired dynamic data of any given bed position. These gaps are introduced at each bed position by the time spent on imaging other bed positions and by scanner system delays due to the time required to translate bed in the axial direction and prepare for the next acquisition of the next position. The result is a sharp reduction in both total counts collected for each axial location as well as in temporal sampling frequency. Furthermore, acquisition of information from fast temporal changes in the early phase of the dynamic study are compromised, as those are not sampled adequately for all bed positions. Consequently, parameter estimates from dynamic whole-body PET acquisitions are potentially compromised by the above limitations in acquisition, degrading their precision and accuracy.
Recently, scanners with increased A-FOV have been developed offering increased sensitivity and substantially more or total-body coverage, that enables synchronous dynamic imaging of the whole body without the need of multiple bed positions and the associated issues with temporal gaps. ~\cite{Karp2020,Siegel2020, Cherry2018}.
These systems are currently found in very few pilot PET centres around the wold, and are not widely adopted in clinics yet. Since the first Total-Body scanner came online there has been an increased focus in the research for clinical applications of such systems, one of which is the use for dynamic whole-body imaging, and as such the research interest in this field is expected to be increased in the next years, with particular focus on clinical utility as well as practical aspects of ease of use and cost-effectiveness.

\subsection*{Challenges and Contributions}
PET dynamic data can be used to extract kinetic parameters over regions of interest or on the voxel level, with the later being used to create parametric images of the applied dynamic model.
Parametric images can provide information that is helpful in identifying and separating out different regions that exhibit different dynamic behaviour, without imposing predefined VOIs in the analysis~\cite{Gallezot2019}.  
One of the major challenge in dynamic whole-body acquisitions using multiple whole-body passes is the estimation of accurate parametric images. Generation of parametric images from dynamic data requires fitting of the dynamic model of interest on time activity curves (TAC) for every voxel in the image. Due to the poor statics and high noise associated with TAC measurements at the voxel level, which are further degraded by DWB acquisition protocols, parametric image estimates can be heavily corrupted by noise and potentially biased. 
The main objective of this thesis was to explore acquisition optimization and novel reconstruction strategies for improvement of whole-body parametric maps. 
The contributions can be separated into two major parts. The first part A) is focused in optimization of the DWB protocol on a clinical PET/MR scanner, aiming in the reduction of temporal gaps in the DWB acquisition and increase of collected counts as well as increase of sampling frequency. The second part B) is focused on the improvement in the use of acquired DWB PET data by exploiting dynamic reconstruction algorithms. As part of this PhD project, various dynamic reconstruction method algorithms were implemented in an open source reconstruction platform, for use with simulated and real data acquired form clinical scanners. An innovative approach for direct multi-bed dynamic reconstruction was developed and applied on real DWB data, for the direct reconstruction of parametric images and temporal regularisation of the reconstructed activity image data.

\subsection*{HYBRID-ITN}
\Gls{hybrid} is an industrial academic \gls{itn}, funded by the European Union's Horizon 2020 research and innovation programme under the Marie Sk\l{}odowska-Curie grant agreement No 764458.
The aim of the \gls{hybrid} group is to advance and fully exploit the potential of integrated, dual-modality and multi-parametric imaging offered by multi-modality hybrid imaging. Quantitative and parametric imaging is at the centre of focus of the group. 

The main challenges that were addressed by the group define the three main work packages of the programme:
\begin{itemize}
    \item WP2: Data collection \\
    The aim of the data collection group is to address challenges of multi-modality data acquisitions for the reliable formation of parametric and multi-parametric images. 
    \item WB3: Data processing \\
    The aim of the data processing group is to explore and improve on methods of extracting information using multi-parametric imaging and multiple parameters. 
    \item WB4: Clinical translation \\
    The aim of this group is to explore integration techniques of multi-parametric imaging into clinical use-case scenarios. 
\end{itemize}

This PhD project was conducted as part of the data collection group WP2 (WP2.4). The aim of the project was to expand computation and use of fully-quantitative parametric maps in PET to whole-body datasets. Specifically the direction of the project was towards development of clinically viable schemes for parametric whole-body PET/MR imaging for improved characterisation of oncological diseases throughout the body. 

Secondments between the partner organisations were planned within the \gls{hybrid} network. The aim of secondments is for the participating PhD students to learn from practices in other organisation and engage in multi-centre research projects. Two academic and one industrial secondment was planned for each student. 
In this project, secondments were planned with the University Medical Center Groningen (UMCG), the Medical University of Vienna (MUW) and with General Electric healthcare (GEHC). These were conducted with the following order, with the described motives and outcomes.

\begin{itemize}
    \item UMCH \\
    This secondment was planned in order to gain experience in PET kinetic modelling for research studies and explore uses of analysis methods that can translate in practices to this PhD project. 
    \item MUW \\
    The initial purpose of this secondment was to make use of already developed advanced reconstruction techniques from this project, to a cohort of epilepsy dynamic PET dataset at MUW for generation of parametric images. Due to the current pandemic, the majority of the secondment was conducted remotely. The collaboration project was slightly adapted to be performed by distance and focused more on motion correction aspects for dynamic brain PET. 
    \item GEHC: Milwaukee \\
    The secondment was planned at the main factory facility which hosts the research and development team of GE's PET/MR systems. The purpose of the secondment was to .
    In addition this secondment served as an experience of research and development in an industrial setup.
    \item GEHC: Zürich \\
    The secondment with GE also took part at the University hospital of Zürich, where GE conducts clinical research and development of protocols. 
\end{itemize}

Apart from the secondments, the \gls{hybrid} network conducted many meetings (approximately twice a year) where all the participants would come together and discuss progress. These meetings served substantially in exchange of ideas and guidance on individual research projects as well as in forming of collaborations. 



\subsection*{Organization of the manuscript}
The manuscript is organised into two major parts, part \textbf{\RNum{1}} outlines the methodologies that are involved for the understanding of the work carried out in this project in four separate chapters which cover PET physics, pharmacokinetics, PET reconstruction theory and implementations in custom software respectively. Part \textbf{\RNum{2}} of the manuscript provides the contributions of this project in three chapters.

Chapter~\ref{Chap3_1:AcquisitionOptimization} describes the development of a custom fully-automated protocol for dynamic whole body imaging on the Signa PET/MR and show results using this protocol on a \gls{nhp} scan.

Chapter~\ref{Chap3_2:SimStudy} describes the development and implementation of novel reconstruction methods within the CASToR software that were subsequently used in this project. The rest of the chapter outlines in detail an extensive simulation study conducted for the evaluation of the developed reconstruction methods for DWB parametric imaging. This simulation study was submitted for publication to the
\textbf{Physics in Medicine \& Biology} journal. 

Chapter~\ref{Chap3_3:IsotoPK} presents the application of the developed reconstruction methods to dynamic whole body data from a first in man pharmacokinetic study conducted our centre. The results of this application were presented in the \textbf{European Association of Nuclear Medicine} 2020 conference. Finally the rest of the chapter provides the results of the application of adaptive modeling in dynamic reconstruction for DWB imaging, that was presented to the \textbf{IEEE/MIC} 2020 conference. 

Finally secondary contributions made in collaboration projects as part of the ITN programme are shown in the end of the manuscript in the Secondary Contributions section. 