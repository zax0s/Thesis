This PhD project was funded and conducted within the \Gls{hybrid} \gls{itn}. This is an industrial and academic \gls{itn}, funded by the European Union's Horizon 2020 research and innovation programme under the Marie Sk\l{}odowska-Curie grant agreement No 764458.
The aim of the \gls{hybrid} group is to advance and fully exploit the potential of integrated, dual-modality and multi-parametric imaging offered by multi-modality hybrid imaging. Quantitative and parametric imaging is at the centre of focus of the group. 

The main challenges that were addressed by the group define the three main work packages of the programme:
\begin{itemize}
    \item WP2: Data collection \\
    The aim of the data collection group is to address challenges of multi-modality data acquisitions for the reliable formation of parametric and multi-parametric images. 
    \item WB3: Data processing \\
    The aim of the data processing group is to explore and improve on methods of extracting information using multi-parametric imaging and multiple parameters. 
    \item WB4: Clinical translation \\
    The aim of this group is to explore integration techniques of multi-parametric imaging into clinical use-case scenarios. 
\end{itemize}

This PhD project was conducted as part of the data collection group WP2 (project WP2.4). The planned direction of the project was defined towards expanding computation and use of fully-quantitative parametric maps in PET to whole-body datasets. Specifically, the direction of the project was set towards the development of clinically viable schemes for parametric whole-body PET/MR imaging for improved characterisation of oncological diseases throughout the body. The carried out project focused on general methodological advancements for whole-body parametric imaging, with applications on real data focusing on whole-body pharmacokinetic studies rather than clinical studies in oncology due to the limited availability of clinical data.

Specific tasks and goals have been pre-defined by the \gls{itn}, to be delivered throughout the project in the form of reports of deliverables and milestones. Those are not included in the thesis manuscript explicitly but many of them form part of the contributions. 

Secondments between the partner organisations were also planned within the \gls{hybrid} network. The aim of secondments is for the participating PhD students to learn from practices in other organisation and engage in multi-centre research projects. Two academic and one industrial secondment was planned for each student. 
In this project, secondments were planned with the University Medical Center Groningen (UMCG), the Medical University of Vienna (MUW) and with General Electric Healthcare (GEHC).
Additionally, the secondment of a student from King's College London (KCL) was organised to take place in our centre. 
The secondment with the student from KCL and the secondment to MUW resulted in collaboration projects that gave which have been produced in two publications. These are attached at the end of this manuscript in the \textbf{Secondary Contributions} section. 

\begin{itemize}
    \item GEHC: Waukesha, WI (3 weeks) \\
    The secondment was planned at the main factory facility which hosts the research and development team of GE's PET/MR systems. The purpose of the secondment was to conduct research for the development of a fully automated dynamic whole-body acquisition protocol for the Signa PET/MR system.
    In addition this secondment served as an experience of research and development in the industrial setup.
    \item GEHC: Zürich (1 week) \\
    The secondment with GE also took part at the University hospital of Zürich, where research and testing is performed for clinical applications of dynamic PET imaging, in order to get an insight into the clinical implementation of dynamic PET protocols.
    \item UMCG (2 weeks) \\
    This secondment was planned in order to gain experience in PET kinetic modelling for research studies and explore uses of analysis methods that could be used for this PhD project. 
    \item MUW (2 weeks) \\
    The initial purpose of this secondment was to make use of already developed advanced reconstruction techniques from this project, to a cohort of epilepsy dynamic PET dataset at MUW for generation of parametric images. Due to the pandemic, the majority of the secondment was conducted remotely. The collaboration project was slightly adapted to be performed by distance and focused more on motion correction aspects for dynamic brain PET.
    
\end{itemize}

Apart from the secondments, the \gls{hybrid} network conducted many meetings (approximately twice a year) where all the ITN partners would come together and discuss progress. These meetings served substantially in exchange of ideas and guidance on individual research projects as well as in forming of collaborations. 

