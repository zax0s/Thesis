\textbf{Introduce the problematique of the thesis/project}

By exploiting the properties of radionuclides, nuclear medicine makes use of compounds which are labelled with radionuclides for diagnostic and therapeutic use in medicine. The general process involves the injection of the radiolabelled compound to the bloodstream , from where it distributes to the entire body via blood circulation. The uptake, metabolism and excretion activity of the injected compound is governed by its type in combination with the underlying tissue physiology and pathology. 
The use of such compounds can be made for therapeutic as well as diagnostic purposes. In this project the focus is on diagnostic uses of radionuclides and more specifically for diagnostic applications in positron emission tomography. 
The basis of use of radionuclies for diagnostic use lies in the emission of gamma rays from natural decay of radionucles within the body that are directly or indirectly detected outside of the body via an imaging device. Their detection provides the information required to generate images of the distribution of the tracer within the body. The imaging techniques can be separated to planar gamma imaging, \gls{spect} and \gls{pet}. For planar gamma single emission photons are used for the formation of images. In \gls{spect} a collection of planar images from multiple angles is used to reconstruct the distribution of the tracer within the body in 3D. 
\Gls{pet} is based on a more specific decay scheme and interactions, the positron emission and annihilation events of positrons and electrons which results in the characteristic emission of two "back-to-back" 511-kev gamma rays that travel in opposite directions. Their detection forms the basis of PET imaging. 
\\
\textbf{To be Continued ...}

