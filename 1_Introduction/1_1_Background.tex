\textbf{Introduce the problematique of the thesis/project}

By exploiting properties of natural decay of radionucles, nuclear medicine makes use of compounds which are labelled with radionuclides for diagnostic and therapeutic use in medicine. The general process involves the injection of radiolabeled compounds into the bloodstream, from where they distribute to the entire body via blood circulation. The uptake, metabolism and excretion activity of an injected compound is governed by the interactions of the compound in the body tissues, in combination with the underlying tissues physiology and pathology. 
The use of such compounds can be made for therapeutic as well as diagnostic purposes. In this project the focus is on diagnostic uses of radionuclides and more specifically for diagnostic applications in \gls{pet}.
The labelled compounds are refereed to as \textit{radiopharmaceuticals}, or more specifically for diagnostic uses as \textit{radiotracers}.
The basis of use of radionuclies for diagnostic use lies in the emission of decay products within the body that are directly or indirectly detected outside of the body via an imaging device. Their detection provides the information required to generate images of the distribution of the tracer within the body. The imaging techniques can be separated to planar gamma imaging, \gls{spect} and \gls{pet}. For planar gamma single emission photons are used for the formation of images. In \gls{spect} a collection of planar images from multiple angles is used to reconstruct the distribution of the tracer within the body in 3D. 
\Gls{pet} is based on a more specific decay scheme and interactions, the positron emission and annihilation events of positrons and electrons which results in the characteristic emission of two "back-to-back" 511-kev gamma rays that travel in opposite directions. Their detection forms the basis of PET imaging. 
\\
With the appropriate corrections, \gls{pet} is able to provide quantitative images of accurate activity concentration. By capturing events over time, information of the dynamic distribution of the tracer can be made. 

