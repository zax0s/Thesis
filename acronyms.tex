%\thispagestyle{plain}
%\chaptermark{Abbreviation}
%\addcontentsline{toc}{chapter}{Abbreviations} \noindent
%\renewcommand{\nomname}{List of Abbreviations}

%--- Acronyms -----------------------------------------------------------------%
% \acrodef{label}[acronym]{written out form} % acronym syntax
%\acrodef{etacar}[$\eta$ Car]{Eta Carinae}   % acronym example
%--- Acronyms -----------------------------------------------------------------%
% how to use acronyms:
% \ac = use acronym, first time write both, full name and acronym
% \acf = use full name (text + acronym)
% \acs = only use acronym
% \acl = only use long text
% \acp, acfp, acsp, aclp = use plural form for acronym (append 's')
% \acsu, aclu = write + mark as used
% \acfi = write full name in italics and acronym in normal style
% \acused = mark acronym as used
% \acfip = full, emphasized, plural, used
%--- Acronyms -----------------------------------------------------------------%
%\chapter*{List of Abbreviations}
%\begin{acronym}
%        \acro{pet}[PET]{Positron Emission Tomography}
%\end{acronym}

\newacronym{pet}{PET}{Positron Emission Tomography}
\newacronym{spect}{SPECT}{Single-photon Emission Computed Tomography}
\newacronym{lor}{LOR}{Line of Response}
\newacronym{hybrid}{HYBRID}{Healthcare Yearns for Bright Researchers for Imaging Data}
\newacronym{itn}{ITN}{Initial Training Network}
\newacronym{pmt}{PMT}{Photomultiplier Tube}
\newacronym{sipm}{SiPM}{Silicon Photomultiplier}
\newacronym{fov}{FOV}{Field of View}
\newacronym{ct}{CT}{Computed Tomography}
\newacronym{mr}{MR}{Magnetic Resonance }
\newacronym{fbp}{FBP}{Filtered Back Projection}
\newacronym{castor}{CASToR}{Customizable and Advanced Software for Tomographic Reconstruction}
\newacronym{psf}{PSF}{Point Spread Function}
\newacronym{sop}{SOP}{Standard Operating Procedure}
\newacronym{nsclc}{NSCLC}{Non-Small-Cell Lung Carcinoma}
\newacronym{mip}{MIP}{Maximum Intensity Projection}
\newacronym{tof}{TOF}{Time of Flight}

% nomenclature:
\newglossaryentry{angelsperarea}{
  name = $a$ ,
  description = The number of angels per unit area,
}
\newglossaryentry{numofangels}{
  name = $N$ ,
  description = The number of angels per needle point
}
\newglossaryentry{areaofneedle}{
  name = $A$ ,
  description = The area of the needle point
}


