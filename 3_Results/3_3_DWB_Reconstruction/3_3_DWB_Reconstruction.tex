\section{Introduction}
Acquisition of DWB datasets in S\&S mode with overlapping bed positions result in complex datasets of PET RAW data, in relation to timing and positional information. Moreover, the addition of an initial DSB acquisition, conducted for IDIF derivation purposes or for the sampling of fast kinetics to allow for more complex kinetic modelling~\cite{Zaker2020}, increases the complexity of the PET RAW dataset. 

Suggested practices~\cite{Karakatsanis2013} and state of the art implementations~\cite{Hu2020} of DWB protocols regard the two datasets (DSB and DWB) as independent and suggest performing independent reconstructions on each dataset. Post-reconstruction kinetic model fitting is then used to combine the TAC information over both datasets. 
Furthermore, the DWB dataset contains locations at the overlapping bed regions which are sampled twice as much as central bed regions, as a result of the overlap sampling necessary to increase sensitivity at the bed position edges. The overlap timing information combines the timing of the adjacent bed positions. In post-reconstruction kinetic modelling, which is applied independently on each voxel TAC, suggested practices make use of modified timing information (average of two bed position timings) for voxels that fall at overlapping regions~\cite{Karakatsanis2013} which result in a degree of degradation of the timing information. Another suggestion is to make use of independent dynamic reconstruction of each bed position from the DWB dataset, as it was conducted in the previous chapter, and apply post-reconstruction overlapping of parametric images ~\cite{Karakatsanis2016a}. In this case if the DSB data are to be included in the dynamic reconstruction, the placement of the DSB bed position needs to be identical to one of the bed positions of the multi-bed dynamic study.

It is important to note that DWB acquired in CBM mode, although free of the need of overlapping bed positions still require similar considerations to be taken into account for each axial location of the DWB acquisition~\cite{Karakatsanis2016b,Hu2020}.

The flexibility offered by the CASToR reconstruction platform that allows for direct reconstruction of multi-bed data~\cite{Ross2004} can be expanded to direct multi-bed dynamic reconstruction of DWB data, using the exact timing information of the dataset and all bed positions within the same iterative loop. Furthermore, the offered flexibility can allow for the use of both DSB and DWB datasets within one unique reconstruction loop, with no constrains in the positioning of the DSB acquisition.
In this chapter, we describe how this novel DWB protocol reconstruction concept was implemented in CASToR and present results from a dynamic reconstruction of real DWB data from the IsotoPK study, which were also presented in the EANM2020 conference~\cite{chalampalakis2020EANM}.


\section{Methods}
\subsection{Dynamic Whole Body Datasets}
A diagram of axial position against time for an example three bed position DWB protocol is shown in figure~\ref{fig_3_3:OverlapFraming}, to demonstrate the differences in temporal sampling between central locations and locations at the bed overlapping regions. The bed locations in this example are identical to the setup used for the NHP study on the Signa PET/MR described in chapter~\ref{Chap3_1:AcquisitionOptimization}.
As it can be seen at the left diagram of figure~\ref{fig_3_3:OverlapFraming} axial positions that fall outside of the overlap region result in framing that is equal to the framing of their respective bed positions. Positions over the overlap regions, seen at the right diagram of figure~\ref{fig_3_3:OverlapFraming}, result in framing equal to that of both adjacent bed positions, which gives twice the number of frames compared to locations outside the overlap. Locations at the overlap regions are not sampled continuously from both contributing bed positions due to the interruption by the movement of the bed and system delays.  
Finally, complexity is increased when the DSB data are included, as shown in figure~\ref{fig_3_3:CompleteProtocolFraming}, with that information contributing to different locations of the DWB acquisition that might fall in overlap regions or not depending on the protocol setup. 

\begin{figure} [ht!]
\centering
\includegraphics[scale=0.50,angle=0]{3_Results/3_3_DWB_Reconstruction/figures/OverlapTiming.pdf}
\caption{Axial slice number vs time for an example three bed positions DWB acquisition. Central axial slices (left) and slices at the overlapping regions (right) considered for the result framing.} 
\label{fig_3_3:OverlapFraming}
\end{figure} 

\begin{figure} [ht!]
\centering
\includegraphics[scale=0.28,angle=0]{3_Results/3_3_DWB_Reconstruction/figures/CompleteProtocolTiming.pdf}
\caption{Axial slice number vs time for an example three bed positions DWB acquisition, including an initial DSB phase of 180 s.}
\label{fig_3_3:CompleteProtocolFraming}
\end{figure} 

If considering only the DWB dataset, and each individual S\&S acquisition as an independent frame, the total number of frames in the DWB dataset will be defined by all the individual S\&S steps.
But not all frames need to be considered for all regions of the effective FOV. 
The choice of the appropriate framing for each location, whether within or outside the overlap, can be made using a mask image that indicates which locations are sampled on each frame. This type of mask could be used for example with post-reconstruction parametric imaging, applied at each voxel to mark which frames need to be considered.

With CASToR the use of multi-bed data can be made directly within a single iterative loop, using the methodology described in section~\ref{chap2_4:MultiBedRecon}. The same framework can be extended to direct multi-bed dynamic reconstruction of DWB datasets, using the same iterative loop. 
For linear dynamic models, dynamic reconstruction (without the use of nested optimization) can be applied directly next to the system matrix, as shown in equation~\ref{eqn:4DMLEM}, within a single iterative loop over all dynamic data $y_{ti}$. 

For DWB data, incorporation of the bed offset information in the projection operation can naturally lead to direct multi-bed dynamic reconstruction, where the selection of the sampled time points for the location of a voxel $j$ is conducted by the projection operation. In this case, the projection operation and consequently the system matrix elements will depend on the frame index number $t$. 
This can also be expressed as an extension of the direct multi-bed reconstruction, described by equation~\ref{eqn2_4:MLEM_multibed}, to dynamic reconstruction where the axial offset of each bed position is incorporated in the time-dependent information of the system matrix.
This provides
%
\begin{equation}
\theta_{pj}^{(k+1)} = \frac{\theta_{pj}^{(k)}}
{\sum_{t=1}^{n_t} B_{tp} \sum_{i=1}^{n_i} P_{tij}} 
\sum_{t=1}^{n_t} B_{tp}  \sum_{i=1}^{n_i} P_{tij} 
\frac{y_{ti}}
{\sum_{d=1}^{n_j} P_{tid} \sum_{q=1}^{n_p} B_{tq}\theta_{qd}^{(k)} + b_{ti} } \\, \\
\label{eqn:4DMLEM_Multibed}
\end{equation} 

where the set of basis functions $B$ is precomputed for all time frames of the DWB acquisition, but is effectively applied in each voxel $j$ for the time frames $t$ that result to non-zero contribution by the back-projection operation ( $\sum_{i=1}^{n_i} P_{tij} \neq 0$).
It is important to note that with this unique iterative loop there is no need for additional considerations after reconstruction of the overlapping operation and regions. The result parametric images $\boldsymbol\theta$ have dimensions of the effective FOV, having accounted for the overlap spatial sensitivity and timing information within the reconstruction.

Use of the same framework to perform individual frame reconstructions of DWB data is also possible, by setting the matrix $\boldsymbol{B}$ to be the identity matrix of size equal to the number of total frames. 
An example from the use of this framework for frame reconstructions is shown in figure~\ref{fig_3_3:Macaque} for a group of three consecutive frames from the \gls{nhp} study of chapter~\ref{Chap3_1:AcquisitionOptimization}. The sensitivity images for the same frames, as estimated by the back-projection operation $\sum_{i=1}^{n_i} P_{tij}$, are shown in figure~\ref{fig_3_3:Macaque_Sensitivity}.
The sensitivity images clearly show the sampled locations per frame and the overlapping locations for adjacent frames. 
Individual frame reconstruction of DWB data results directly to images that account for the bed offset in image space, which can subsequently be directly used for post-reconstruction kinetic model fitting without the need for positional considerations. 
Additionally, for post-reconstruction kinetic model fitting, the result sensitivity frame images can be used as a mask for the application of the dynamic model on each frame and voxel combination.

%
\begin{figure} [ht!]
%\centering
\includegraphics[scale=0.5,angle=0]{3_Results/3_3_DWB_Reconstruction/figures/Macaque_3D.pdf}
\caption{Example three frame images from a three bed DWB acquisition (NHP study of chapter~\ref{Chap3_1:AcquisitionOptimization})} 
\label{fig_3_3:Macaque}
\end{figure} 
%
\begin{figure} [ht!]
%\centering
\includegraphics[scale=0.42,angle=0]{3_Results/3_3_DWB_Reconstruction/figures/Macaque_Sensitivity.pdf}
\caption{Example three frame sensitivity images from a three bed positions DWB acquisition.} 
\label{fig_3_3:Macaque_Sensitivity}
\end{figure} 
%
In addition to DWB data alone, the described methodology for direct multi-bed dynamic reconstruction allows for any sequence of the acquired bed/frames to be used within the iterative reconstruction loop. Furthermore, the use of the DSB dataset can also be included in the iterative loop, regardless of its axial position, while also being sub-divided into multiple short frames.

\subsection{Dynamic reconstruction: nested optimization for DWB datasets}
As described previously, dynamic reconstruction is often performed using the nested optimization framework, to accelerate convergence~\cite{Wang2010,Matthews2010}.
Some additional considerations need to be made before the use of this multi-bed dynamic reconstruction framework with nested optimization.

The nested optimization framework decouples the tomographic update process over the PET data from the dynamic model fitting process on image space. The two steps are conducted respectively with an MLEM update over the PET raw data using equation~\ref{eqn:EM_Update_image} to get an individual EM update image for each frame, followed by kinetic model fitting which optimises the likelihood function of equation~\ref{eqn:NestedOptimization}. 

For single bed dynamic data the sensitivity term of the likelihood function, $\big[\sum_{i=1}^{n_i} P_{ib}\big]$ of equation~\ref{eqn:NestedOptimization}, can be ignored assuming that it is constant over time for each voxel $b$, and the likelihood function can be optimised using an image based EM update over image space with equation~\ref{eqn:NestedEM}~\cite{Wang2010,Reader2014}.
It is important to note that if dead-time corrections, necessary for accurate quantitative reconstruction, are included in the system matrix then the sensitivity term needs to be kept and accounted for in the nested optimisation. But in most applications of nested optimisation on single-bed dynamic data in the literature this term is ignored.

In the case of DWB data and overlapping bed positions, the sensitivity term $\big[\sum_{i=1}^{n_i} P_{tij}\big]$ has to be maintained, as its value will change with different frames for voxels $j$ in the overlapping regions.
Thus the two step process, for direct multi-bed dynamic reconstruction with nested optimization, can be written as

\begin{equation}
\label{eq3_3:NestedMultibed}
\text{}
\begin{cases}  
f_{tj}^{(\textrm{EM})}(\bm\theta^{(k)}) = \frac{f_{tj}(\bm\theta^{(k)})}{\sum_{i=1}^{n_i} P_{tij}} 
\sum_{i=1}^{n_i} P_{tij} 
\frac{y_{ti}}{\sum_{d=1}^{n_j} P_{tid} f_{td}(\bm\theta^{(k)}) + b_{ti} } \\ \\
\bm{\theta}^{(k+1)} = \argmax{\bm{\theta}} 
\sum_{t=1}^{n_t} \sum_{j=1}^{n_j} \left[ \sum_{i=1}^{n_i}  P_{tij} \right]
\left[ -f_{tj}(\bm\theta) + 
ln( f_{tj}(\bm\theta)) 
f_{tj}^{(\mathrm{EM})}(\bm{\theta}^{(k)})
\right] .\\
\end{cases}
\end{equation}

The tomographic update of this two-step optimization process is effectively an update over the DWB data for independent frame reconstruction, followed by an image space optimization process that now needs to consider the sensitivity $\big[\sum_{i=1}^{n_i} P_{tij}\big]$ of each time point $t$ in the TAC of each voxel $j$. For linear models, where before an image based MLEM algorithm was used for this image space optimization process, a weighted MLEM update can be used with the sensitivity of each time point $t$ as the weight. Alternatively, for linear and non-linear models, WLS based optimization algorithms can be used with the sensitivity value used as the weight. Both of these options were implemented in CASToR, with results shown in this chapter were produced by the use of the nested EM approach.
%
\iffalse 
\begin{equation}
\theta_j^{(k+1)} = \frac{\theta_j^{(k)}}
{\sum_{p=1}^{n_p} w B_{pj}}
\sum_{p=1}^{n_p} w B_{pj} 
\frac{f_{j}^{(EM)}(\bm{\theta}^{(k)})}{\sum_{d=1}^{n_j} B_{pd}\theta_d^{(k)} } \\.
\label{eqn3_3:WeightedNestedEM}
\end{equation}
\fi
%
\subsection{Real Data}
Two DWB scans from the IsotoPK study were used for assessing the described direct multi-bed dynamic reconstruction method. Both scans were conducted on a single volunteer in a single day, without and with the use of the inhibitor (rifampicin) respectively. The two scans will be referred to as \textit{CTRL} and \textit{RIF} scans, standing for control and rifampicin. %The volunteer was a male (25y) weighing 66 Kg on the examination day.
The two dynamic scans were conducted with the injection of 141.53 MBq and 90.77 MBq of [$^{11}$C]Glyburide respectively. The first scan was conducted with 14 WB passes, with 9$\times$20, 5$\times$30 s frames per bed position. The second scan included 15 WB passes, with 9$\times$20, 6$\times$30 s frames per bed position. Both scans begin with a 180 s DSB acquisition centred over the liver, starting at the time of injection, before the DWB acquisition. By splitting the DSB into 18x10 s frames and considering each Step and Shoot acquisition as an individual frame, the two scans resulted in a total of 88 and 93 frames respectively. 
The planned bed positions are shown in figure~\ref{fig_3_3:IsotoPK_BedPositionsOnMR} for the DSB and DWB phase of the \textit{CTRL} scan. 

The complete \textit{CTRL} and \textit{RIF} datasets, including the DSB and DWB data, were reconstructed within the developed direct multi-bed dynamic reconstruction platform, using an OSEM algorithm with 28 subsets and EM nested optimization of 20 sub-iterations. Both datasets made use of TOF information within the reconstruction. 
As the IsotoPK study is an exploratory pharmacokinetic study, dynamic reconstructions were performed with the spectral model for temporal regularisation of frame activity estimates as well as for the direct estimation of $K_1$ parametric images, without imposing strong assumptions about the underlying kinetics.
Additional dynamic reconstructions were performed for DWB data, without the use of the initial DSB data, to test extrapolation of the spectral model on early (non-sampled) frames. 
A total number of 17 spectral basis functions ($M=$ 16) were used, with $\beta_1 ... \beta_{M-1}$ logarithmically spaced within the range of 3 to 0.001 $min^{-1}$, while including $\beta_0$ and $\beta_{M}$ to account for trapping and the blood fraction in the data. Similarly to the simulation study, the assumption of the total blood activity concentration $C_{B}$ being proportional to the arterial blood plasma $C_{P}$ was used in the estimation of the basis functions.
Manual arterial blood samples were taken during both scans, to derive the input function and for blood analysis (to measure the metabolites fraction and for plasma binding of the tracer). The measured input functions were linearly interpolated and then used in the estimation of the spectral basis functions. 
Initial results on the study have shown very little metabolic activity of Glyburide, which did not require modelling, allowing for the assumption of constant ratio of arterial to total blood to be made.

Using the fitted spectral model, the approximate $\boldsymbol{K_1^*}$ parametric images were estimated similar to the methodology used in the simulation study of chapter~\ref{Chap3_2:SimStudy}.
In addition to the dynamic reconstructions, individual frames (3D) reconstruction was performed using the same framework and OSEM algorithm with 28 subsets.

\begin{figure} [ht!]
\centering
\includegraphics[scale=0.42,angle=0]{3_Results/3_3_DWB_Reconstruction/figures/3_3_IsotoPK_CTRL_PositionsOnMR.pdf}
\caption{Planned DSB and DWB bed positions shown on coronal MRAC image, with bed start (\protect\tikz[baseline]{\protect\draw[line width=0.5mm] (0,.8ex)--++(1,0) ;}) and end (\protect\tikz[baseline]{\protect\draw[line width=0.5mm,densely dashed] (0,.8ex)--++(1,0) ;}) positions.} 
\label{fig_3_3:IsotoPK_BedPositionsOnMR}
\end{figure} 

The VOIs listed in table~\ref{tab:IsotoPK_VOIs} were drawn manually and used to validated and compare the reconstruction methods.

\begin{table}[ht!]
\centering
\caption{\label{tab:IsotoPK_VOIs} VOIs used for evaluation of DWB scans.}
\begin{tabular}{lll}
\toprule
\textbf{VOI Name} & \textbf{Availability in Data}  \\
\midrule
Brain        & DWB                 \\
Myocardium & DSB \& DWB              \\
Left Ventricle (LV) & DSB \& DWB     \\
Left Kidney (Kidney$_\mathrm{L}$) & DSB \& DWB  \\
Right Kidney (Kidney$_\mathrm{R}$) & DSB \& DWB \\
Spleen & DSB \& DWB \\
Liver  & DSB \& DWB \\
Aorta & DSB \& DWB \\
Bladder & DWB \\
Leg Muscle & DWB \\
\toprule
\end{tabular}
\end{table}

\subsubsection{Liver dual-input function simulation}
For the liver which is one of the main organs of interest in the IsotoPK study, as discussed in section~\ref{liver_PV_theory}, the dual input function model is necessary for accurate modelling of tracer behaviour. 
When this is not accounted for, parametric estimates derived from single input function models are expected to be biased.  
A toy simulation was conducted to study how quantification of $K_1^{*}$ is affected by the use of the spectral model with a single input function and to study whether relative differences of true underlying $K_1$, such as those expected from the comparison of the \textit{CTRL} to the \textit{RIF} scan, can still be accurately deduced using the spectral model's estimated $K_1^{*}$. This short simulation study is presented in appendix~\ref{chap:AppendixC}.

\section{Results}
\subsection{Comparison between 3D and 4D spectral reconstruction on DWB data}
Coronal \gls{mip} images of 3D reconstructions from the \textit{CTRL} scan are shown in figure~\ref{fig_3_3:IsotoPK_CTRL_DSB_3D} and figure~\ref{fig_3_3:IsotoPK_CTRL_DWB_3D}, for a late frame of the DSB acquisition and early frames of the DWB acquisition respectively. 
Similarly, \gls{mip} images of 4D spectral reconstruction of the \textit{CTRL} scan are shown in figure~\ref{fig_3_3:IsotoPK_CTRL_DSB_4D} and figure~\ref{fig_3_3:IsotoPK_CTRL_DWB_4D} for the same respective frames. 
As the spectral model is applied to the image space of the entire effective FOV, the 4D reconstruction results in frame images that are estimated from the fitted model on the entire FOV. As such, activity estimates of locations that are not sampled for a given frame within the DWB scan are interpolated from the frame data of the entire DWB scan. For the frames of the DSB scan, locations that are not sampled by the single bed acquisition have activity estimates that are extrapolated from the fitted spectral model of later frames.

\begin{figure} [ht!]
\centering
\includegraphics[scale=0.52,angle=0]{3_Results/3_3_DWB_Reconstruction/figures/3_3_IsotoPK_CTRL_DSB_3D.pdf}
\caption{Coronal MIP image of 3D reconstruction (4it28sub) of a single frame from the DSB acquisition on the \textit{CTRL} scan}
\label{fig_3_3:IsotoPK_CTRL_DSB_3D}
\end{figure} 

\begin{figure} [ht!]
\centering
\includegraphics[scale=0.52,angle=0]{3_Results/3_3_DWB_Reconstruction/figures/3_3_IsotoPK_CTRL_DWB_3D.pdf}
\caption{Coronal MIP images of 3D reconstructions (4it28sub) of a single frame per bed position from the first whole-body pass of the DWB acquisition on the \textit{CTRL} scan.}
\label{fig_3_3:IsotoPK_CTRL_DWB_3D}
\end{figure} 

\begin{figure} [ht!]
\centering
\includegraphics[scale=0.52,angle=0]{3_Results/3_3_DWB_Reconstruction/figures/3_3_IsotoPK_CTRL_DSB_4D.pdf}
\caption{MIP image of a single frame from 4D reconstruction (15it28sub) of the CTRL scan}
\label{fig_3_3:IsotoPK_CTRL_DSB_4D}
\end{figure} 

\begin{figure} [ht!]
\centering
\includegraphics[scale=0.52,angle=0]{3_Results/3_3_DWB_Reconstruction/figures/3_3_IsotoPK_CTRL_DWB_4D.pdf}
\caption{MIP image of the first whole-body pass frames from 4D reconstruction (15it28sub) of the CTRL scan.}
\label{fig_3_3:IsotoPK_CTRL_DWB_4D}
\end{figure} 

Plots of mean VOI activity against iteration, for early and late frames over the liver and the leg muscle VOI were used to evaluate at which iteration the mean values begin to converge. These plots are shown in figure~\ref{fig_3_3:IsotoPK_CTRL_DWB_4D_Convergence} and figure~\ref{fig_3_3:IsotoPK_CTRL_DSB_3D_Convergence} of the appendix~\ref{chap:AppendixC}. For 4D spectral reconstruction, the evaluated mean values showed relatively stable behaviour from the 15th iteration onwards, which was chosen for subsequent comparisons of reconstructions. For 3D reconstruction mean values did not show stable behaviour in both regions within the first 8 iterations and the 4th iteration was chosen in subsequent comparisons as a comprise between the reliability of mean values convergence and image noise.  

TACs of the evaluated VOIs are plotted in figure~\ref{fig_3_3:IsotoPK_CTRL_DWB_4D_vs_3D_Central} and~\ref{fig_3_3:IsotoPK_CTRL_DWB_4D_vs_3D_Peripheral}, for VOIs sampled in both DSB and DWB acquisitions and for VOIs sampled only during the DWB acquisition respectively. In figure~\ref{fig_3_3:IsotoPK_CTRL_DWB_4D_vs_3D_Peripheral} the time point of the start of the DWB acquisition is indicated by the red dotted line.
As already mentioned, the 4D spectral reconstruction results in an activity estimate of every location of the image for every frame of the study, while 3D reconstruction results in activity estimates for the sampled frames only at each location. This is why the TACs from the 4D reconstruction of the \textit{CTRL} data show a total of 88 frame points for all VOIs, while for 3D reconstruction a total of 32 frame points (18 from DSB plus 14 from DWB) is seen for regions included in the DSB acquisition and 14 frame points for regions seen solely by the DWB acquisition. 

For the VOIs seen by both DSB and DWB acquisitions, the TACs between 3D and 4D spectral reconstruction are in good agreement for all evaluated VOIs, as seen in~\ref{fig_3_3:IsotoPK_CTRL_DWB_4D_vs_3D_Central}. For the brain and muscle VOIs seen by the DWB acquisition alone, there is a relatively good agreement, with slight overestimation in early frames of the brain region and underestimation on later frames as seen in~\ref{fig_3_3:IsotoPK_CTRL_DWB_4D_vs_3D_Peripheral}. By contrast, at the bladder VOI there is a strong mismatch of the 3D and 4D spectral reconstruction, with the 3D reconstruction showing clearly the bladder filling process and the 4D spectral reconstruction showing an increasing behaviour produced by the exponential functions. 
The VOI means of the spectral parametric images are given in the appendix material, where it is seen that the component of full trapping $\phi_0$ is strongly used in the bladder region, in an attempt to fit the underlying process. But the use of the input function and the decaying exponentials is not enough to fit this process. Further considerations to this issue are addressed in the section~\ref{sub_section:residuals}.

\begin{figure} [ht!]
\centering
\includegraphics[scale=0.5,angle=0]{3_Results/3_3_DWB_Reconstruction/figures/3_3_IsotoPK_CTRL_DWB_3D_vs_4D_central.pdf}
\caption{VOI mean time activity curves for 3D and 4D spectral reconstruction. VOI regions shown which are included in both DSB and DWB acquisition.}
\label{fig_3_3:IsotoPK_CTRL_DWB_4D_vs_3D_Central}
\end{figure}

\begin{figure} [ht!]
\centering
\includegraphics[scale=0.5,angle=0]{3_Results/3_3_DWB_Reconstruction/figures/3_3_IsotoPK_CTRL_DWB_3D_vs_4D_peripheral.pdf}
\caption{VOI mean time activity curves for 3D and 4D spectral reconstruction. VOI regions shown which are covered only by the DWB acquisition, whose start time is designated with a vertical dotted line.}
\label{fig_3_3:IsotoPK_CTRL_DWB_4D_vs_3D_Peripheral}
\end{figure} 

\subsection{Comparison of 4D Spectral reconstruction on DWB data, with and without the use of DSB data}
A comparison between 4D spectral reconstructions, with and without the use of the initial DSB data, is made in figure~\ref{fig_3_3:IsotoPK_CTRL_DWB_4D_vs_4D_Central}, for VOIs covered by both DSB and DWB acquisitions. These show a good agreement of the two spectral reconstructions, with a consistent slight error in the early DWB frame estimations of the reconstruction without the use of DSB data. This error is relatively small compared to the mean activity of each VOI (maximum of 12.7\% error seen in Spleen VOI). A very good agreement is seen on the liver between the two reconstructions.

The extrapolation of the fitted spectral model from DWB data alone can be made for the early non sampled frames. These show a sharp drop of the fitted TACs, indicating the inadequacy of the used data for extrapolation of the fast early frame behaviour and the cause for the systematic mismatch at early frames of the DWB acquisition.
In particular, the extrapolated model is unable to reproduce the early peak, as seen clearly in the ventricle, the myocardium and the spleen, which indicates that separation between the exchange components and the blood fraction component of the spectral model is compromised when the early dynamic data are not available.

\begin{figure} [ht!]
\centering
\includegraphics[scale=0.5,angle=0]{3_Results/3_3_DWB_Reconstruction/figures/3_3_IsotoPK_CTRL_DWB_4D_vs_4D_central.pdf}
\caption{VOI mean time activity curves for 4D spectral reconstructions, with and without the use of DSB data. VOI regions shown are included in both DSB and DWB acquisitions.}
\label{fig_3_3:IsotoPK_CTRL_DWB_4D_vs_4D_Central}
\end{figure} 

\subsection{Direct WB parametric image estimation from 4D Spectral reconstructions}
Using the set of equations~\ref{eqn:AllSpectralEqns} and specifically the equation for the derivation of $K_1$, the parametric maps of $K_1^{*}$ were produced by summation of basis $(\phi_0-\phi_{M-1})$. Similar to the simulation study before, the blood fraction correction was neglected in favour of avoiding voxel-to-voxel division and induction of excessive image noise. The result parametric maps for both \textit{CTRL} and \textit{RIF} scans are shown in figure~\ref{fig_3_3:IsotoPK_K1_MIP} and figure~\ref{fig_3_3:IsotoPK_K1_SingleSlice}, as a \gls{mip} in the coronal plane and as a single coronal slice showing the liver, the kidneys and the spleen.

Mean VOI $K_1^{*}$ values, shown in figure~\ref{fig_3_3:IsotoPK_K1_drop}, show a considerable reduction from the \textit{CTRL} to the \textit{RIF} scan by the administration of the inhibitor. This drop reflects the difference between specific and non-specific uptake of the novel Glyburide tracer in the various imaged organs, with the strongest apparent difference seen in the liver.

\begin{figure} [ht!]
\centering
\includegraphics[scale=0.5,angle=0]{3_Results/3_3_DWB_Reconstruction/figures/3_3_IsotoPK_K1_MIPs.pdf}
\caption{$K_1$ MIP images from the 4D spectral reconstruction using both DSB and DWB acquisitions, shown for the \textit{CTRL} and \textit{RIF} scan.}
\label{fig_3_3:IsotoPK_K1_MIP}
\end{figure} 

\begin{figure} [ht!]
\centering
\includegraphics[scale=0.5,angle=0]{3_Results/3_3_DWB_Reconstruction/figures/3_3_IsotoPK_K1_SingleSlice.pdf}
\caption{$K_1^{*}$ values as estimated from the 4D spectral reconstruction for VOI regions included in both DSB and DWB acquisitions , shown for the \textit{CTRL} and \textit{RIF} scan}
\label{fig_3_3:IsotoPK_K1_SingleSlice}
\end{figure} 

\begin{figure} [ht!]
\centering
\includegraphics[scale=0.5,angle=0]{3_Results/3_3_DWB_Reconstruction/figures/3_3_IsotoPK_K1_drop.pdf}
\caption{$K_1^{*}$ values as estimated from the 4D spectral reconstruction for VOI regions included in both DSB and DWB acquisitions , shown for the \textit{CTRL} and \textit{RIF} scan}
\label{fig_3_3:IsotoPK_K1_drop}
\end{figure} 

The results of a short dual input simulation study, given in appendix~\ref{chap:AppendixC}, indicate that accurate quantification of $K_1$ (or $K_1^{*}$) in the liver using only the arterial input function as an input is not achievable as expected. 
But the simulation showed that a relative comparison of spectral model $K_1^{*}$ estimates reflects approximately linearly the differences of the true underlying $K_1$, without the explicit need for modelling the dual input function.
Further investigations are needed in the use of the spectral model within the liver, starting from simple mathematical formulations of the problem before testing further using PET simulated data or the real data. In any case, knowledge or valid assumptions on the rate of dispersion and the ratio of portal to arterial blood in the liver can be advantageous as they would reduce the number of unknown parameters. Furthermore, joint estimation approaches in the estimation of parameters in the liver could be assessed for uses in WB dynamic studies.

\subsection{Fit errors and error propagation}
\label{sub_section:residuals}
As seen in figure~\ref{fig_3_3:IsotoPK_CTRL_DWB_4D_vs_3D_Peripheral}, the mismatch between the 3D and 4D spectral reconstructions at the bladder TAC is the strongest from all the evaluated VOIs. The source of this error is the inability of the spectral model to fit the bladder filling process, by any combination of the decaying exponentials convolved with the arterial input function. This mishmash is of particular concern for dynamic reconstruction as it can be the source of errors that propagate spatially during the dynamic reconstruction process over the entire field of view~\cite{Kotasidis2014c,Kotasidis2016a}. 
This issue is addressed further in the work presented in chapter~\ref{Chap3_4:Residual} by use of adaptive residual modelling for dynamic reconstruction.

\section{Discussion}

We have developed and presented a framework for direct multi-bed dynamic reconstruction of DWB data, that allows for synchronous use of both DSB and DWB PET raw data, typically acquired in dynamic studies over the whole body~\cite{Karakatsanis2013}. 
This framework can be used for dynamic reconstruction by making use of the accurate positional and timing information of all the individual bed raw PET data within a single iterative loop, performed either directly within a single system matrix or using the nested optimization approach.
In this study, we have tested this framework with real DWB data, from a first in man pharmacological study, where the spectral model was used within the reconstruction to avoid having to assume and enforce a specific compartmental model.
The result (temporally regularised) activity images show estimates with a smooth transition in the axial direction across the effective FOV, in terms of image activity values and image noise. Although we hypothesised that use of the data within a single iterative loop results in improved estimates at the overlap region, compared to the state of the art methods that perform parametric image overlapping after reconstruction~\cite{Karakatsanis2016a,Hu2020}, we do not directly compare with these methods. To address this comparison, we performed an evaluation study for these methods using the NHP dataset of chapter~\ref{Chap3_1:AcquisitionOptimization}, but results were inconclusive and showed the need for a detailed simulation study to assess differences in terms of bias and noise properties of the overlapping regions against known ground truth simulated values. 

The IsotoPK real DWB data used in this study included a 180 s DSB acquisition centred over the liver prior to the DWB acquisition. This has enabled the estimation of $K_1^{*}$ parametric images over the region covered by both DSB and DWB acquisitions. As shown before in chapter~\ref{Chap3_2:SimStudy}, that initial information is crucial for the estimation of $K_1$. Thus, although the use of the DSB initial phase allows for adequate sampling of fast kinetics and estimation of fast kinetics sensitive parameters in DWB studies, it still limits the estimation for parametric images to an axial coverage of a single bed position. All other regions outside this bed coverage were sampled with a 180 s delay, which as shown in chapter~\ref{Chap3_2:SimStudy} produces erroneous estimates of $K_1^{*}$. 
In this study this relates to $K_1^{*}$ estimates in regions outside the coverage of the DSB acquisition in figure~\ref{fig_3_3:IsotoPK_K1_MIP} and figure~\ref{fig_3_3:IsotoPK_K1_SingleSlice}. 
Further investigation in the setup of the acquisition protocol is required to quantify the minimum amount of fast kinetics sampling required for reliable deduction of parametric $K_1$ images and other parametric images sensitive to early fast kinetics. An analogous study conducted for the estimation of 2TC micro-parameters in FDG imaging resulted in evaluation of the sensitivity of micro-parameters against a sampling of specific time points~\cite{Zuo2019}. A similar study for the spectral model analysis method can evaluate the need of the DSB phase for estimation of $K_1$ and other parameters and if possible adapt the acquisition for reliable estimation of those parameters over the whole effective FOV. 
Another disadvantage of the need for a DSB acquisition is the uneven sampling over different locations of the effective FOV that results in different noise properties across the axial direction. Equal sampling of all regions is preferred to provide comparable noise properties in the axial direction of the FOV.

There are many challenges and limitations in dynamic reconstruction and parametric imaging, which are of particular concern for DWB imaging where the dynamic model is applied over the whole body~\cite{Gallezot2019}. 
Firstly, a single input function has been used for the estimation of the basis functions of the spectral model which is then applied over the entire effective FOV. The use of a single input function does not account for dispersion and delay effects that are varying for different locations in the body. As kinetic models and in particular the spectral model rely heavily on the input function, any errors or inaccuracies can lead to considerable estimation errors. Ideally, the delay and dispersion properties of the input function need to be estimated for each region, which would then be used for calculating a set of spatially variant spectral basis functions.
Furthermore, the input function was derived from manual samples whose sampling frequency compromises the measurements of the peak of the input function. This information over the peak is of significant importance for the estimation of micro-parameters. 
Methods for joint-estimation of dispersion and delays or joint estimation of the input function itself within the reconstruction process have been previously proposed~\cite{Wang2013,Reader2014}, but so far have only been applied in DSB studies~\cite{Reader2014} and studies using total-body data~\cite{Feng2019} where sensitivity and sampling-frequency are much higher compared to multi-bed DWB imaging.
Many considerations are needed to evaluate the translation of these methods in multi-bed DWB, where the poorer data statistics and sampling frequency might not allow for reliable estimation of all the involved parameters of these joint estimations. 
Image derived input functions can also be used as an alternative to manual samples, but they also suffer from limitations regarding sampling frequency over the input function peak and are normally measured from a few locations without accounting for delay or dispersion. 

Moreover, regarding the application of the input function over the effective FOV, certain organs might not follow well the single input function models. 
In this study, a major organ of interest is the liver, which is supplied with blood via two different routes. We showed using a simple simulation that although the spectral model is able to fit the liver TAC relatively well, the result $K_1^{*}$ parameters were strongly biased. Accurate quantification requires estimation of both input functions. Recently joint estimation approaches~\cite{Wang2018} and image derivation approaches~\cite{Wang2021} have been shown to produce good results in DSB studies. Their translation to DWB and generalisation for use with other tracers requires further research. Nevertheless, this example of the liver shows again that in the quantification of parameters there is a need for spatially variant modelling. Still, the use of a generic model such as the spectral model has the ability to adequately fit the underlying tracer behaviour and provide temporally regularised activity estimates.

Another considerable limitation, that has not been addressed in this study, is subject voluntary and involuntary motion. Inevitably these types of motion will be present in dynamic studies, especially when imaging the whole body with DWB that involve frequent bed movements. In dynamic reconstruction these effects need to be accounted for within the reconstruction process, to avoid image artefacts and modelling errors.
But when considering the whole-body, there are many types of complex motion taking place during the duration of a dynamic study. These include respiratory motion, cardiac motion, head movements, bowel motion and other internal organ movements, etc~\cite{Gallezot2019}. Estimation of all these types of motion is a complex task. Many data driven~\cite{Schleyer2015,Kesner2019,Lu2020} and joint estimation approaches~\cite{Bousse2017,Jiao2017} have been proposed for motion estimation and correction, but their use for complex elastic motion is still actively researched. Recently machine learning approaches have been proposed to improve on elastic motion estimation from low statistic gated images~\cite{Zhou2020,Zhou2021}. For complex motion detection over the whole-body, the use of histo-images (COD images), which is constantly improving with TOF hardware advancements, is actively being researched for use in motion detection/estimation~\cite{Panin1551}. 
Nevertheless, the task of motion detection from dynamic PET data is more complex when considering a dynamically changing tracer distribution rather than a static tracer distribution, which is assumed in most approaches discussed above.
Specifically for PET/MR imaging, the use of specialised MR sequences has been previously suggested and successfully used for complex elastic motion estimation~\cite{Tsoumpas2010,Catana2015}, but these motion detection MR sequences can be time consuming. In DWB imaging where delays between individual bed position acquisitions have a strong impact on the estimation of kinetic parameters, there can be a great impact on PET parametric image quality by the sacrifice of sampling frequency due to additional MR sequences on each bed position. 

\section{Conclusion}
A working direct multi-bed dynamic reconstruction method was presented and tested using real data from a DWB pharmacokinetic study. Evaluation tests for dynamic spectral reconstruction against regular frame by frame reconstruction of the data showed relatively good agreement of VOI TAC data in all regions except the bladder, where the filling process was badly modelled. 
Finally, there is a need for accurate motion estimation and correction within the dynamic reconstruction process that is imperative for artefact free and error free parametric image estimations over the whole body.
Estimation of all types of complex motion over the whole body, using ideally data driven approaches, needs to be addressed further for DWB acquisitions.