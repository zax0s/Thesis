In this chapter, we discuss the complications arising from kinetic model fitting errors in dynamic reconstruction, for DWB imaging and WB parametric maps. We present our work towards the reduction of these errors using adaptive residual modelling for dynamic reconstruction, applied on the IsotoPK CTRL dataset presented in the previous chapter. The focus of this application was the model fitting mismatch over the bladder region and any potential propagating errors. To our knowledge, this is the first time a method of this type has been applied to real DWB data.

The results of this work were presented in the \textbf{IEEE/MIC} 2020 conference. This chapter is adapted from the conference record of this submission.
\newpage
\section{Introduction}
Dynamic PET has been traditionally utilised in single organ studies, but recently there has been increased interest in whole body dynamic imaging, for research and for potential future clinical applications~\cite{Lammertsma2017,Leahy2018,Rahmim2019,Fahrni2019}. Synchronous dynamic PET of the whole body requires scanner geometries that can encompass the entire length of the human body. Although systems capable of whole body axial coverage have been recently developed~\cite{Cherry2017}, their availability is not yet widespread. Current clinical scanners achieve whole body coverage using multiple axial bed positions or acquisition protocols with continuous axial bed motion. Based on these acquisition modes whole body dynamic protocols have been developed that make use of repeated whole body passes~\cite{Karakatsanis2013}. 
Conventional analysis of dynamic data is performed with independent 3D reconstructions of frame data, followed by kinetic model fitting. Whole-body dynamic protocols make the task of reconstruction and subsequent kinetic modelling more challenging due to the introduction of large temporal gaps in the data which can lead to biased estimates and increased noise, especially for parametric maps. 
To improve image quality and account for the noise in the raw data, 4D reconstruction algorithms can be used to directly incorporate the dynamic model of interest in the reconstruction~\cite{Reader2014}.
Depending on the choice of the dynamic model, these reconstructions can directly provide the parameters of interest or temporal regularisation on the reconstruction process of frame data. The spectral analysis model, inspired by the homonym method~\cite{Cunningham1993}, fits on any underlying kinetic behaviour that can be described by compartmental modelling. When used in 4D reconstruction it provides temporal regularisation as well as direct estimation of some kinetic parameters. Use of this model has shown to reduce noise and bias of kinetic parameter estimates when applied on data from whole body dynamic protocols~\cite{Chalampalakis2019}.

When considering the effective field of view of whole body dynamic studies, certain regions might not fulfil the generic assumptions of the spectral analysis model. For example processes such as delay of tracer delivery and non-arterial tracer delivery (ex. bladder filling).
In 4D reconstruction, the tomographic update process is entangled with the dynamic model and errors from the temporal model fit have the potential to spatially propagate to other regions. Due to the fact that raw data from multi-bed whole body dynamic studies are acquired and reconstructed as separate beds, the potential for errors propagation is limited in the bed position where they originate from. It has also been shown that use of time of flight (TOF) information in the reconstruction aids in reducing spatial propagation of errors~\cite{Kotasidis2016a}. Nevertheless, model fit errors have the potential to propagate and when possible should be corrected or accounted for. For this reason, when 4D reconstruction is used and especially when applied on whole body datasets extra consideration has to be given on improving the model fit for all regions.

The application of an adaptive residual model as a second dynamic model has been previously suggested for reducing model fit errors~\cite{Matthews2012}. Its application in 4D reconstruction has been shown to reduce bias at the cost of increased noise for direct estimation of kinetic micro-parameters~\cite{Kotasidis2014c}.
The objective of this work was to implement and evaluate adaptive residual modelling in 4D reconstruction, using real whole body dynamic PET data, with no prior knowledge of the underlying kinetics and the distribution of residuals. For this purpose, the spectral analysis model was used in conjunction with a residual model in 4D reconstruction to maintain genericity in the reconstruction with minimum assumptions on the underlying kinetics. 
The end objective was to assess if adaptive residual modelling can be permanently applied in 4D reconstruction, with no prior knowledge of kinetics and model fit errors, by relying on the selectivity of the algorithm for correction of underlying model fit errors, when and where those arise. 

\section{Methods}
\subsection{Data acquisition}
Dynamic whole body (DWB) data using a novel ${}^{11}$C-Glyburide tracer were acquired from a healthy volunteer on the GE Signa PET/MR, using a dedicated DWB protocol of 5 bed positions to provide sufficient coverage of the body. The data were acquired as part of the exploratory pharmacological study IsotoPK which is conducted in our centre for studying the distribution of OATP transporters in the body and their role in the delivery of drugs to tissues~\cite{Marie2019}. In this work a control scan was used (\textit{CTRL} scan from chapter~\ref{Chap3_3:IsotoPK}).
In detail the dynamic acquisition consisted of a single bed acquisition centred over the liver, imaged for 3 minutes from tracer injection, followed by 14 whole body passes of 9x20s and 5x30s frames for each bed position. 
Arterial blood samples were collected manually during the whole study, to derive the input function.
\subsection{Reconstruction}
The spectral analysis model was used in all 4D reconstructions, using 6 spectral basis functions ($M=$ 5) with $\beta_1 ... \beta_{M-1}$ logarithmically spaced within the range of 3 to 0.001 $min^{-1}$, while including $\beta_0$ and $\beta_{M}$ to account for trapping and the blood fraction in the data. The 4D reconstruction algorithm with adaptive residual modelling was developed using the spectral model as primary model and an adaptive secondary model, similar to the proposed method by Matthews et al.~\cite{Matthews2012}. 
 
The steps of the adaptive residual modelling 4D reconstruction algorithm are:

\begin{enumerate}
  \item Start with image estimate $\lambda_{tj}^{(k)}$, for voxel $j$ and frame $t$ at iteration $k$. 
  \item Perform tomographic update using MLEM to get the EM update images $f_{tj}^{(EM)}$.
  \item Perform primary model fit using the NNLS algorithm on $f_{tj}^{(EM)}$ and estimate model parameters $\bm{\theta}_{j}^{(k+1)}$.
  \item Calculate residuals as $ r_{tj} = f_{tj}^{(EM)} - f_{tj}^{(spectral)}(\bm{\theta}_{j}^{(k+1)}) $.
  \item Perform PCA analysis on residuals to estimate a set of residual basis functions (RBF) which form the secondary (adaptive) model. 
  \item Estimate modelled residuals $g_{tj}$ from the LS fit of the secondary model on the $r_{tj}$ data. 
  \item Estimate optimal fraction $K_{j}$ for residuals re-introduction, using generalised cross validation.
  \item Add fraction of modelled residuals in image estimate $\lambda_{tj}^{(k+1)} = f_{tj}^{(spectral)}(\bm{\theta}_{j}^{(k+1)}) + K_{j} g_{tj} $ and repeat from step 2, 
\end{enumerate}

where NNLS stands for Non-Negative Least Squares and $f_{tj}^{(spectral)}(\bm{\theta}_{j})$ is the spectral model which provides activity estimates for voxel $j$ at frame $t$ given model parameters $\bm{\theta}_{j}$.
        
In step (5) of our adaptive modelling implementation, the residual data were spatially smoothed with an isotropic Gaussian kernel of 32mm FWHM, to enhance the PCA analysis results. Smoothing of the residuals was not however applied in the fitting process of the secondary model in step (6). A fixed number of the most significant PCA components was used as basis functions of the adaptive residual model. The components were updated on each iteration of the 4D algorithm.

For the estimation of the optimal fraction $K_{j}$, the following equation from Matthews \textit{et al.}~\cite{Matthews2012} was used, which is derived from generalised cross-validation,

\begin{equation}
K_{j} = \frac{1-\frac{r_{df}}{r_{RSS}}}{1-r_{df}} , 
\end{equation}

where $r_{df} = \frac{n_s}{n_f - n_p}$, with $n_p$ and $n_s$ being the number of parameters of the primary and secondary model respectively, and $r_{RSS}$ being the fraction of modelled to measured residuals sum of squares. 

The dynamic data from each bed of the study were reconstructed as individual frames using 3D MLEM (150it), using 4D reconstruction (300it) and with the developed adaptive algorithm 4D-ResidMod (300it).
No subsets were used for the reconstructions performed in this work.
The adaptive reconstruction algorithm was evaluated using the 2 and 3 most significant PCA components as RBFs, referred to as 4D-ResidMod-2RBF and 4D-ResidMod-3RBF respectively. We also implemented a version of the adaptive algorithm with the addition of temporal Gaussian smoothing of 62s FWHM applied on the residual data used for PCA analysis in step (5) of the algorithm, referred as reconstruction 4D-ResidMod-3RBF-Tempfilt.

All reconstructions were performed with the open source reconstruction platform CASToR~\cite{Merlin2018}, using resolution modelling and TOF information. Correction for motion has not been included in the reconstructions. PCA analysis was performed using the Eigen library~\cite{eigenweb}. 

\section{Results}
As seen in Fig.~\ref{fig:Frame14Differences} (top row) by the difference in activity estimates from 3D and 4D reconstruction in a late frame, the 4D algorithm with the spectral model alone resulted in lower activity estimates over the bladder region and higher activity estimates in some large blood vessels at the shoulders and pelvic regions. No noticeable differences were seen in other areas, apart from random variations in activity due to differences in noise levels of the compared reconstructions.

\begin{figure} [!ht]
\centering
\includegraphics[scale=1.15, angle=0]{3_Results/3_4_Residual/figures/Frame14DifferencesV2.png}
\caption{Coronal view of activity distribution from the evaluated reconstructions (frames of 41.2 to 44.1 minutes post injection). Difference images are also shown for selected cases against the 3D reconstruction.}
\label{fig:Frame14Differences}
\end{figure}

As expected, the filling process of the bladder seen in the bladder time activity curve from 3D reconstruction in Fig.~\ref{fig:TACs} (bottom row) cannot be modelled by the spectral model because the filling process is not directly relatable to arterial tracer delivery at the local (voxel) level. Differences in some large blood vessels could potentially be attributed to delay of the input function, which is not accounted for in the spectral model where a single input function is assumed to apply for all regions. 

Reconstruction of the data with adaptive residual modelling resulted in improved estimation of the bladder filling process, for both options of residual modelling using 2 and 3 residual basis, as it can be seen in the difference images of Fig.~\ref{fig:Frame14Differences} (middle row) and the bladder TAC in Fig.~\ref{fig:TACs}. 
No improvements were seen for the differences at the blood vessels of the shoulders and pelvic regions. 

At the liver, where most of the tracer uptake is concentrated, 4D and 3D reconstruction TACs show a subtle mismatch between 1100 and 1500 seconds (Fig.~\ref{fig:TACs}, top row). Analysis on the liver dome VOI (Fig.~\ref{fig:TACsFilt}, top row) clearly shows this difference, which originates from underlying respiratory motion that has been captured in the PET dynamic data. 4D reconstruction alone was unable to model these fluctuations, as expected, but the adaptive algorithm using 3 residual basis has achieved to model for this behaviour as seen in Fig.~\ref{fig:TACsFilt}.
Since physiological kinetic processes, whether sufficiently modelled or not, are not expected to result in sudden changes of activity between adjacent frames, we implemented a version of the algorithm with temporal smoothing prior to PCA analysis to prevent from modelling changes due to fast processes such as respiratory motion. On this case this algorithm (4D-ResidMod-3RBF-Tempfilt) has managed to maintain the selective adaptive modelling behaviour over the bladder while avoiding fitting the motion related effects over the liver dome as seen in Fig.~\ref{fig:TACsFilt}. 

\begin{figure} [!htbp]
\centering
\includegraphics[scale=0.64,angle=0]{3_Results/3_4_Residual/figures/TACs.png}
\caption{Average and standard deviation time curves of the liver and bladder VOIs, for the comparison of 3D to 4D reconstruction results as well as 4D reconstruction with residual modeling using different number of residual basis functions.} 
\label{fig:TACs}
\end{figure}

\begin{figure} [!htbp]
\centering
\includegraphics[scale=0.74 ,angle=0]{3_Results/3_4_Residual/figures/TACsFilt.png}
\caption{Average and standard deviation time curves of the dome of the liver the whole liver and bladder VOIs, for the comparison of 4D reconstruction with residual modeling with and without temporal filtering of residuals.} 
\label{fig:TACsFilt}
\end{figure}

In all regions of the effective FOV the addition of adaptive modelling has resulted in the increase of image noise, compared to 4D reconstruction, with the higher number of residual basis resulting in higher noise. The introduction of temporal filtering in the adaptive model estimation process has aided in reducing the induction of noise for the majority of the evaluated regions in our study. 
A comparison of the effect on image noise across regions within this whole body study can be made by the VOIs standard deviation shown in Fig.~\ref{fig:BarPlot} for all evaluated algorithms.
An example map of the fraction $K_{j}$ and the modelled residuals $g_{jf}$ is given in Fig.~\ref{fig:FractionMap}. These maps demonstrate the desired selectivity of the algorithm, concentrated over the bladder in this case, but also the
spill of noise from the residual space in these maps which is subsequently added in the reconstruction process.

\begin{figure} [!htbp]
\centering
\includegraphics[scale=0.44 ,angle=0]{3_Results/3_4_Residual/figures/FractionMap.png}
\caption{Fraction map of $K_j$ and modelled residuals $g_{jf}$ of the residual model 4D-ResidMod-3RBF-Tempfilt at iteration 75.} 
\label{fig:FractionMap}
\end{figure}

\begin{figure} [!htbp]
\centering
\includegraphics[scale=0.65 ,angle=0]{3_Results/3_4_Residual/figures/BarPlot.png}
\caption{Standard deviation of VOIs for regions in the whole-body, for all evaluated reconstructions.} 
\label{fig:BarPlot}
\end{figure}

\section{Discussion}
General dynamic models such as the spectral analysis model can be useful in 4D reconstruction of whole body dynamic data using new tracers, as they do not impose strong assumptions about the underlying kinetics. But over the effective field of view of whole body dynamic studies certain regions and their physiological processes might not fulfil the general assumptions of the model. In these cases, the adaptive residual modelling strategy can be utilised to improve parameter estimates for these regions and help in minimising the risk of spatial propagation of errors to other well modelled regions.

In this work, 4D reconstruction with the spectral analysis model and adaptive residual modelling was applied on real whole body dynamic data from a first in man study of a new tracer. The algorithm identified and selectively corrected the image activity estimates over the poorly modelled bladder region, by modelling the residuals of that region. This resulted in a 4D reconstruction that did not suffer from the model fit errors over the bladder and potential errors propagated to other regions originating from the bladder region. 
The use of the adaptive model as a secondary model increased the complexity of the modelling in the reconstruction process and resulted in an increase of image noise, with higher number of residual basis functions resulting in higher image noise as expected and inline with previous observations in a simulation study~\cite{Kotasidis2014c}.
In our work we saw that noise from the residual space propagates in the modelled residuals and the optimum fraction values of the secondary model, seen in Fig.~\ref{fig:FractionMap}, and is subsequently re-introduced in the reconstruction process. 

Motion will unavoidably always be present in the raw data of dynamic studies and when it is not accounted for in the 4D reconstruction process will give rise to model fit errors. In our evaluation we saw that the adaptive algorithm attempted to model residuals caused by respiratory motion, when the level of complexity of the secondary model permitted that, even though this was not its original intended use.
As any tracer's kinetic processes are expected to be slower than variations due to respiratory motion, pre-treatment of residuals with temporal filtering in the residual model estimation step aided in averting the adaptive model from fitting these motion induced differences while maintaining its capability of modelling kinetic processes. In addition, temporal filtering also aided in reducing the increase in noise from the use of the secondary model.

In this work we conclude that the use of the secondary adaptive model based on PCA analysis can be a practical solution to spectral analysis model fit errors, when added to the generic 4D reconstruction algorithm used with whole body dynamic data from new tracers. The combination of spatial and temporal filtering in the treatment of residual data prior to PCA analysis provided the optimal results in our tests, in terms of algorithm selectivity and levels of added noise. 

In this work, we presented the application of a previously suggested method for adaptive residual modelling in dynamic reconstruction to reduce errors that arise from poorly modelled regions. As the spectral model is quite generic in the assumptions of underlying model behaviour, by only assuming a generic compartmental model structure and a single input function, the use of a generic adaptive modelling approach can be included in the reconstruction to model processes that do not fall in these assumptions while maintaining overall genericity. 
Although spatial and temporal filtering in pre-treatment of residual data successfully avoided the fitting of motion induced residuals, these types of residuals should ideally not be considered as model errors since they involve motion of the areas described within each voxel's TAC. Motion correction can be easily integrated in the dynamic reconstruction with adaptive modelling, and should be performed when reliable motion estimations are available. Nevertheless, temporal filtering did reduce the induction of noise by the residual modelling process and thus can still be considered for general use even with motion corrected data.
Another limitation in adaptive modelling within multi-bed dynamic reconstruction is that each bed position is sampled at different time points. In our implementation PCA analysis was performed in a bed-by-bed basis, which enforced the use of individual bed dynamic reconstruction. Subsequently, for the application presented in the included conference record, the reconstructed images of each bed were combined after reconstruction using weighted averaging. Further considerations are required for extracting residual information over different regions of the DWB scan, with attention to structures giving rise to residuals that fall in the overlap regions, to enable the use of this approach with direct multi-bed reconstruction. Segmentation approaches on residual space, previously proposed for residual modeling~\cite{Germino2016}, could be employed for the extensions of residual modelling in direct multi-bed dynamic reconstruction.

\section{Conclusion}
To improve on estimation over the bladder, an adaptive residual model was used in dynamic reconstruction of DWB data, which showed promising results of selectivity and modelling of the errors. Further work is required to enable the use of data-driven adaptive modelling approaches for direct multi-bed reconstruction on the DWB setup. 

