\chapter*{Conclusions and prospects}
The objective of this thesis was to explore and assess methods for improving whole-body parametric imaging, for clinical and research pharmacological applications, using multi-bed dynamic whole body PET/MRI data.
As hybrid PET/MR scanners and the majority of other PET systems provide only a limited axial field-of-view (FOV), dynamic whole-body (DWB) protocols are used to extend the effective FOV over the required axial length. But these acquisitions come at the cost of limitations in acquisition counts and sampling frequency that degrade parametric image quality and accuracy.
Several aspects were investigated, throughout the process of DWB data acquisition to data processing and reconstruction. We largely focused on the use of dynamic reconstruction algorithms for improving the use of the PET acquired data in the estimation of activity and parametric images.

Specifically, in this project we have
\begin{enumerate}
\item Developed an optimised protocol for S\&S mutli-bed DWB acquisitions on the Signa PET/MR.
\item Evaluated and compared advanced dynamic reconstruction algorithms for whole-body parametric imaging, with focus in oncological imaging.
\item Explored the use of a direct multi-bed dynamic reconstruction framework on a real WB pharmacological study.
\item Evaluated the use of adaptive residual modelling applied in DWB data and a generic dynamic reconstruction algorithm.
\end{enumerate}

Firstly, we have worked closely with the manufacturer of the GE Signa PET/MR in methods for reducing the loss of counts and sampling frequency in DWB imaging by means of reducing delays in the acquisition process. This resulted in a custom fully-automated protocol that has shown to considerably reduce delays when applied on a real DWB study on a \gls{nhp} subject. However, approval for use of this protocol with human studies requires a more thorough review of safety measures and closer collaboration with the manufacturer. Ideally, CBM acquisition techniques should also be considered for implementation as they can contribute further towards improvements and flexibility in acquisition. 

Secondly, we made use of dynamic reconstruction for DWB parametric imaging, with focus on oncology applications of dynamic FDG PET and Patlak parametric imaging, using simulated and real data.
We proposed the use of an indirect method for DWB parametric imaging based on the spectral analysis model and dynamic reconstruction which allows for generic compartmental modelling to be used during reconstruction for temporal regularisation. While the Patlak model is limited to data after steady state conditions have been reached, this novel spectral reconstruction approach enables the use of all dynamic data in the reconstruction and offers greater modelling flexibility. Post-reconstruction Patlak parametric imaging using the regularised data of the spectral reconstruction outperformed direct Patlak dynamic reconstruction. But even this method did result in high parametric image noise when reconstruction was iterated sufficiently to ensure accurate quantification. Potentials for improvement should be examined further using additional regularisation methods to deliver acceptable parametric image noise for clinical applications and ensure accurate quantification.
%All the reconstruction methods used in this work were implemented in the open-source and fully quantitative reconstruction platform CASToR and are available in the current public release of CASToR. 
Finally, we have expanded our application of dynamic reconstruction to real data from an exploratory, first in man, WB pharmacological study. We extended the use of the dynamic reconstruction to direct multi-bed reconstruction, which enabled the use of the complete DWB dataset within a single iterative loop for dynamic reconstruction. This approach inherently handles the individual bed position data's sensitivity and timing information, notably for the bed overlapping positions where previously suggested practices made use of compromising solutions, while also allowing for the use of the DSB phase data independently of its axial position. In this type of applications, the use of spectral reconstruction is favoured as it does not impose strong assumptions on the unknown underlying kinetics. Preliminary results using VOI based analysis showed good agreement with results from 3D regular reconstruction. These applications also enabled the generation of surrogate parametric $\boldsymbol{K_1}$ maps, for relative comparison between scans.
Further comparisons with post-reconstruction parametric imaging need to be conducted to explore the reliability and additional benefits of this reconstruction approach towards accurate parametric imaging, but these were not conducted at this preliminary stage of the study.

In this application, we identified a limitation in the modelling process over the bladder, that had the potential to be the source of spatially propagating errors within the dynamic reconstruction process. 
To minimise this risk we have included an adaptive residual modelling approach within the reconstruction. This method identified and selectively corrected the model activity estimates over the poorly modelled bladder region, which provided a significant reduction of model fit errors at the cost of added noise over the entire FOV. Some optimisation advancements were made using pre-treatment of residual data, but further developments are needed for reducing the induction of noise by this approach. 

A major limitation in the application of dynamic reconstruction on WB data is motion and motion induced artefacts that affect image quality and quantification. 
In the setting of DWB imaging, complex motion exists in the PET data that ideally needs to be accounted for and corrected for within the reconstruction process. 
Many approaches for including motion correction in the reconstruction have been proposed and successfully used. But estimation of the underlying elastic motion vectors over time and over the WB is a challenging task. Moreover, the limited sampling of MR and PET in DWB acquisitions may reduce the sensitivity of many data-driven approaches for motion estimation. In our evaluations, using PET raw data and MR data from attenuation correction sequences, we were unable to use common motion estimation tools for the successful estimation of elastic motion vectors.
Recently numerous Machine Learning (ML) applications for motion estimation and correction have been proposed in medical imaging. This increasing use of ML approaches for motion estimation could lead to further developments that could be useful in the DWB motion estimation problem.

During the course of this project, the introduction of the first Total-Body PET scanner in 2018 has ignited the research interest in DWB imaging. The considerable increase in sensitivity and sampling frequency by synchronous dynamic total-body scans resulted in research projects that span from clinical applications of parametric imaging to joint estimation of complex parameters over the whole-body. These also include innovative methods using the PET TOF information for motion detection and correction.
But the availability of these systems is limited, due to the very high cost and no exclusive total-body clinical applications at this time.

More recently, extended FOV scanners became available by commercial providers with an axial FOV of approximately one meter. These offer considerably more coverage than current systems of 15-26 cm axial FOV. At this time there are two systems installed in clinics, with more underway. The greater availability of extended FOV (ex-FOV) systems will lead to more research projects of DWB imaging for clinical applications~\cite{Slart2021}. As in clinics the systems with limited FOV will still be used in the near future, there can be many opportunities for interesting comparison studies on WB parametric imaging to identify the limits of multi-bed DWB imaging with respect to synchronous DWB imaging by ex-FOV scanners.
For example, it could be envisaged that for dynamic imaging of slow processes, such as in the case of Patlak FDG, the DWB protocols on regular scanners could provide sufficient parametric image quality for clinical applications and exclusive use of ex-FOV scanners is not necessary. On the other hand, for kinetics sensitive to early fast dynamics synchronous DWB imaging from ex-FOV scanners is expected to provide strong and clear benefits.

There is a considerable research focus on the subject, from whole-body modelling, whole-body corrections and reconstruction approaches, to ML applied in whole-body image analysis. The technological advancements are going to further fuel this growth of interest, but what is left to be seen is which applications of these techniques and technologies will be predominant in future clinical and pharmacokinetic applications.