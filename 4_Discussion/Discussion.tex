\chapter*{Conclusions and prospects}
The objective of this thesis was to explore and assess methods for improving whole-body parametric imaging, for clinical and pharmacokinetic applications, using multi-bed dynamic whole body PET/MRI data.
As hybrid PET/MR scanners and the majority of other PET systems provide only a limited axial field of view, dynamic whole-body (DWB) protocols are used to extend the acquisition over the required axial length. But these acquisitions come at the cost of limitations in acquisition counts and sampling frequency, that degrade parametric image quality and accuracy.
Several aspects were investigated, throughout the process of DWB data acquisition to data processing and reconstruction. We principally focused on the use of dynamic reconstruction algorithms for making better use of the PET acquired data in the estimation of activity and parametric images.

Specifically, in this project we have
\begin{enumerate}
\item Developed an optimised protocol for S\&S mutli-bed DWB acquisitions on the Signa PET/MR.
\item Evaluated and compared novel dynamic reconstruction algorithms for whole-body parametric imaging, with applications in oncological imaging.
\item Tested a direct multi-bed dynamic reconstruction framework on a real WB pharmacokinetic study.
\item Evaluated the use of adaptive residual modeling applied in DWB data and a generic dynamic reconstruction algorithm.
\end{enumerate}

Firstly, we have worked closely with the manufacturer of the GE Signa PET/MR in methods for reducing the loss of counts and sampling frequency in DWB imaging by means of reducing delays in the acquisition process. This resulted in a custom fully-automated protocol that has shown to considerably reduce delays, when applied on a real DWB study on a \gls{nhp} subject. However, approval for us of this protocol with human studies requires a more thorough review of safety measures and closer collaboration with the manufacturer. Ideally CBM acquisition techniques should also be considered for implementation as they can contribute further towards improvements and flexibility in acquisition. 

Secondly, we made use of dynamic reconstruction for DWB parametric imaging, specifically for oncology applications of dynamic FDG PET and Patlak parametric imaging, with simulated and real data. 
We proposed the use of an indirect method for DWB parametric imaging based on the spectral analysis model and dynamic reconstruction which allows for generic compartmental modelling to be used during reconstruction for temporal regularisation. While the Patlak model is limited to data after stead state conditions have been reached, this novel spectral reconstruction approach enables use of all dynamic data in the reconstruction and offers greater modeling flexibility. Post-reconstruction Patlak parametric imaging using the regularised data of the spectral reconstruction outperformed direct Patlak dynamic reconstruction. But even this method did result in high parametric image noise when reconstruction was iterated sufficiently to ensure accurate quantification. Potentials for improvement should be examined further using additional regularisation methods to deliver acceptable parametric image noise for clinical applications and ensure accurate quantification.
All the reconstruction methods used in this work were implemented in the open source and fully quantitative reconstruction platform CASToR and are available in the current public release of CASToR. 

Finally, we have extended our application of dynamic reconstruction to real data from an exploratory, first in man, WB pharmacokinetic study. We extended the use of the dynamic reconstruction to direct multi-bed reconstruction, which enabled the use of the complete DWB dataset within a single iterative loop for dynamic reconstruction. This approach inherently handles the individual bed position data's sensitivity and timing information, notably for the bed overlapping positions where previous suggested approaches made use of some compromises. In this type of applications the use of the spectral reconstruction is favoured as it does not impose strong assumptions on the unknown underlying kinetics. Preliminary results using VOI based analysis showed good agreement with results from 3D regular reconstruction. This applications also enabled the generation of surrogate parametric K1 maps, for relative comparison between scans.
Further post-reconstruction parametric imaging needs to be conducted to further explore the benefits of this reconstruction approach towards accurate parametric imaging, but these were not conducted at this preliminary stage of this study.
In this application we identify an limitation in the modeling process over the bladder, that can be the source of spatially propagating errors within the dynamic reconstruction process. 
To minimise this risk we have included an adaptive residual modelling approach within the reconstruction. This method identified and selectively corrected the image activity estimates over the poorly modelled bladder region, which provided significant reduction of model fit errors over the bladder at the cost of overall added noise. Some optimisation advancements were made using pre-treatment of residual data, but further considerations need to be taken for reducing the induction of noise by this approach. 

A major limitation in the application of dynamic reconstruction on WB data is motion and motion induced artefacts. 
In the setting of DWB imaging, complex motion exists in the PET data that ideally needs to be accounted and corrected for within the reconstruction process. Many approaches for including motion correction have been proposed and successfully used. But estimation of the underling elastic motion vectors over time and over the WB is a challenging task. Moreover the limited availability of MR and PET data over the WB reduces the sensitivity of many data-driven approaches for motion estimation. In our evaluations, attempts using PET raw data and MR data from attenuation correction sequences, we were unable to use common motion estimation tools for successful estimation of elastic motion vectors.
The increasing use of Machine Learning (ML) approaches applied in motion estimation can results in approaches that could be useful in the DWB motion estimation problem.


During the course of this project, the introduction of the first Total-Body scanner in 2018 has ignited the research interest in DWB imaging. The abundance of offered sensitivity and sampling frequency of total-body scans resulted in research projects focusing in many aspects, from clinical applications of parametric imaging to joint estimation of complex parameters over the whole-body. These also include innovative methods using the PET TOF information for motion detection and correction. 
But availability of these systems is limited, due to the very high cost and limited unique clinical applications at this time.

But, the availability of extended FOV scanners by manufacturers can lead to greater availability of ex-FOV system in clinics, that will lead to research project focused in clinical application. As these are expensive systems, new clinical potentials need to be considered to justify their cost, rather than speed-up factors of static imaging or lower dose imaging alone. 

Research from these scanners in clinics, that already make use of regular 15-26 cm a-FOV scanners can lead to interesting comparison studies on WB parametric imaging applications and identify the limits of DWB imaging with respect to the synchronous DWB imaging offered by ex-FOV scanners. 
For late dynamic imaging of slow changes, such as the case of FDG imaging for Patlak modeling, it could be envisaged that DWB protocols on regular scanners could provide sufficient parametric image quality for clinical applications and exclusive need of ex-FOV scanners is not necessary. But in early fast dynamics, the availability of ex-FOV scanner data is definitely expected to provide a strong benefit.

Many other research is focused around the subject, with focus on whole-body modeling, whole-body corrections and reconstruction approaches, to ML applied in whole-body image analysis. The field is growing fast, and the technological advancements are going to further fuel this growth. What is left to be seen is which applications of these techniques and technologies will be the predominant in clinical applications.