% Brief introduction with 
\Gls{pet} is a quantitative imaging technique that makes use of 
positron-emitting radionuclides for the study of biochemical and physiological processes \textit{in vivo}. The molecules of interest for the processes under study are labelled with a radionuclide and then introduced in the body. The PET imaging system provides information about the distribution of the labelled molecules \textit{in vivo} over time and can be used to deduce information about the underlying process kinetics. 

\section{Principles of pharmacokinetic modelling}
Pharmacokinetics refers to the study of absorption, distribution, metabolism and excretion of drugs in living systems. The study of pharmacokinetics is based on measurements of concentration of drugs and their metabolites in tissues over time. Pharmacokinetic models are mathematical relationships, derived from prior knowledge or previous observations, which can be tested against the measurements in order to describe the underlying behaviour. These models describe the transport and binding rates of tracer from local concentration differences across boundaries, that can be either physical (such as a membrane or an organ outline) or conceptual boundaries as for example for example between bound and unbound tracer. These boundaries define separate compartments with distinct activity concentration which form the basis of pharmacokinetic models, also referred to as compartmental models.
%There are three assumptions are made for compartmental modelling to be valid:
%The first is the tracer assumption which requires that the presence of the tracer in PET studies is not influencing the physiological processes and molecular interactions. This is the case in most PET studies as the specific activity of the tracer (Activity/tracer concentration) is kept low. The second assumption is that the physiological processes and molecular interactions are in constant state during the duration of the PET study. The third and final assumption is that tracer is homogeneously distributed within each compartment. 

The three key assumptions underlying compartmental modelling are:
\begin{itemize}
\item The concentration of the PET tracer is not high enough to influence the physiological processes and endogenous molecular interactions under study.
\item That all physiological processes and interactions are in constant state for the duration of the PET study.
\item The assumption that tracer concentration is instantly uniform in all compartments of the model.
\end{itemize}

By common convention in pharmacokinetic modelling, the first compartment is the arterial plasma pool from where the tracer distributes to tissues. The concentration of tracer in the arterial plasma $C_P(t)$ over time $t$ is measured or deduced from population studies and applied to the model, as an input function that powers the system.
If the tracer is metabolised during the study, this process needs to be modelled using metabolite measurements. Only the concentration of parent tracer can be used as input function in quantitative analysis of the tracer kinetics.
By convention in dynamic PET studies conducted under a single session, the time point $t_0=0$ is set to be the tracer injection time. \\
Compartmental models behave according to a set of first-order ordinary differential equations, which means that change of concentration in one compartment is a linear function of the concentrations of all compartments. This linearity establishes that the measured tissue activity concentration will be the convolution of the input function with the impulse response function (IRF) of the system, which is described by the compartmental model and its parameters.
As such the tissue activity concentration $C_T(t)$ can be modelled as
\begin{equation}
 C_T(t) = C_P(t) \ast \textrm{IRF}(t)  \\ , \\ 
\end{equation}
where $\ast$ is the convolution operator. 

When a measurement of activity concentration is made with \gls{pet}, the measurement will include the underlying tissue activity and activity from the intravascular blood in this tissue. The proportion of tissue volume occupied by intravascular blood is $V_B$ is commonly referred to as blood fraction.The measured activity concentration $C_{PET}$ can be expressed as
\begin{equation}
{C_{PET}}(t)  = (1-V_{B}){C_{T}}(t) + V_{B}C_{B}(t) \\ , \\
\label{eqn:CPET}
\end{equation}
where $C_{B}$ is the total blood activity concentration.
Measurements of the %metabolite corrected
arterial plasma to total blood ratio can be made to relate between $C_{B}$ and $C_P(t)$. 
For the tracers of interest in this PhD project, the ratio of the two concentrations is stable and the relationship between the two concentrations is assumed to be $C_{B}(t) = r C_{P}(t)$. 

\section{One-tissue compartment model}
As the simplest model, the one-tissue compartment model (1TCM) can be described using two constant rates for the input and output of tracer from the tissue, $K_1$ and $k_2$ respectively. A representation can bee seen in figure~\ref{fig:1_2TCM}.

\begin{figure}[ht]
	\includegraphics[width=0.8\textwidth]{2_Theory_Methods/figures/TissueCompartmentModels.pdf}
	\centering
	\caption{Representation of compartments and exchanges rates for 1TCM (top) and 2TCM (bottom)}
	\centering
	\label{fig:1_2TCM}
\end{figure}

The rate of change of the activity concentration in tissue $C_T$ will depend on the metabolite corrected arterial plasma input function $C_P$ and the activity concentration in tissue, described by the differential equation~\ref{eqn:1TCM_Diff}.

\begin{equation}
  \frac{\mathrm d C_T(t)}{\mathrm d t} = K_1 C_P(t) - k_2 C_T(t)
  \label{eqn:1TCM_Diff}
\end{equation}
\begin{equation}
   C_T = K_1 e^{-k_2 t} \ast C_P(t) 
  \label{eqn:1TCM}
\end{equation}
\begin{equation}
   C_{PET} =   K_1 e^{-k_2 t} \ast C_P(t) (1-V_B) + C_B(t) V_B
  \label{eqn:1TCM_CPET}
\end{equation}

Equation \ref{eqn:1TCM} is the solution of the differential equation \ref{eqn:1TCM_Diff}, which becomes \ref{eqn:1TCM_CPET} for the PET measurement which includes the fractional blood volume. %In equation \ref{eqn:1TCM_CPET} it is important to note that the blood fraction is related to the whole blood activity concentration $C_B$ , rather than the metabolite corrected activity concentration $C_A$. When the tracer metabolic activity is negligible and under the assumption of negligible tracer delay and diffusion from arterial to venous blood TACs, the two concentrations ($C_A = C_B$) can be set equal in favour if simplified models and estimations. 

\section{Two-tissue compartment model}
The two-tissue compartment model (2TCM) is a commonly used model as it describes the behaviour of ${}^{18}F$-Fluorodeoxyglucose (${}^{18}\mathrm{F}$-FDG), which is commonly used tracer in clinical and research PET protocols. The separation into two tissue compartments is made to distinguish between free and trapped (bound) tracer, represented as $C_F(t)$ and $C_B(t)$ respectively, with the trapping caused by the tracer being metabolised in the cells by mitochondria. The set of differential equations describing the 2TCM is shown in equations~\ref{eqn:2TCM_Diff}.

\begin{subequations}
\begin{align}
%C_{PET} = (C_F + C_B)(1-V_B) + V_B C_A \\
\frac{d}{dt}C_F(t) = K_1 C_A(t) - (k_2 + k_3)C_F(t) + k_4 C_B(t) \\ 
\frac{d}{dt}C_B(t) = k_3 C_F(t) - k_4 C_B(t)  
\end{align}
\label{eqn:2TCM_Diff}
\end{subequations}

The pathway of tracer from the trapped to the free state, via the process of dephosphorylation, is commonly considered negligible for the duration of common PET studies. With the assumption of $k_4=0$ the two differential equations in \ref{eqn:2TCM_Diff} simplify to the system \ref{eqn:2TCM_Diff_k4=0} which when solved results to equation \ref{eqn:2TCM}. 

\begin{subequations}
\begin{align}
\frac{d}{dt}C_F(t) = K_1 C_A(t) - (k_2 + k_3)C_F(t) \\ 
\frac{d}{dt}C_B(t) = k_3 C_F(t)  
\end{align}
\label{eqn:2TCM_Diff_k4=0}
\end{subequations}
%
\begin{equation}
C_T(t) =  C_F(t) + C_B(t) = K_1 ( e^{-(k_2+k_3)t} + \frac{k_3}{k_2+k_3}(1-e^{-(k_2+k_3)t})) \ast C_P(t)   
\label{eqn:2TCM}
\end{equation}

The parameters $K_1$, $k_2$ and $k_3$, which are refereed to as micro-parameters, can be estimated by fitting equation~\ref{eqn:2TCM} on measured \gls{tac} data. 

A parameter of interest for clinical studies, is the influx rate constant $K_i$, given by equation~\ref{eqn:FDG_Ki}. This is considered as a macro-parameter of the system and can be understood as the the proportion of flux $K_1$ that results to trapped tracer. 

\begin{equation}
K_i = \frac{K_1 k_3}{k_2+k_3}
\label{eqn:FDG_Ki}
\end{equation}


\section{Gjedde-Patlak linearisation method}

Direction estimation of model micro-parameters, such as those of~\ref{eqn:2TCM}, requires non-linear fitting optimization procedures. These are commonly time consuming and their estimations are susceptible to noise.
For parametric imaging, where the model estimation has to be performed for every voxel of the image, estimation of micro-parameters is commonly avoided due to the poor statics and high noise associated with \gls{tac} measurements at the voxel level. 
Linearisation methods allow for transformation of the measured data, to enable use of linear least-square fitting procedures for estimation of model macro-parameters, under certain assumptions. Furthermore these methods reduce the number of parameters to be estimated, thus reducing the variability of estimates and sensitivity to noise. 

With the assumption of irreversible tracer behaviour the Gjedde-Patlak method has been developed, described by Gjedde~\cite{Gjedde1982} and Patlak \textit{et al.}~\cite{Patlak1985}. The proposed transformation is

\begin{equation}
\label{eqn:PatlakModel}
\frac{C_{T}(t)}{C_{P}(t)} = K_i \frac{\int_{0}^{t} C_{P}(\tau) d\tau}{ C_{P}(t)} + V_{\alpha}   \ , \;  t>t_{ss} \ ,
\end{equation}

where $K_i$ is the steady state trapping rate and $V_{\alpha}$ the apparent volume of distribution. The transformation is valid for time points $t$ after steady state conditions are achieved at time $t_{ss}$. Steady state conditions are achieved when the reversible compartments are in steady-state equilibrium with the plasma blood compartment. 

The Gjedde-Patlak linearisation method, refereed simpler as \textit{Patlak model}, is commonly used with the 2TCM model for FDG, under the assumption of $k_4=0$. In this case the Patlak model parameters can be related to micro-parameters using equation~\ref{eqn:FDG_Ki} and 

\begin{equation} 
V_a  = \frac{K_1 k_2}{(k_2+k_3)^2} \\ . \\ 
\end{equation}

%For a PET measurement the observed activity $C_{PET}(t)$ described by equation~\ref{eqn:CPET}.
With the assumption of $C_{B}(t) = r C_{P}(t)$, we can substitute equation~\ref{eqn:PatlakModel} into equation~\ref{eqn:CPET} and describe the observed activity $C_{PET}(t)$ using the Patlak model as

\begin{equation} 
{C_{PET}}(t)  = \underbrace{(1-V_{B})K_i}_{\theta_1} \int_{0}^{t} C_{P}(\tau) d\tau +  \underbrace{V_{\alpha}+r V_{B}}_{\theta_2} C_{P}(t) \\ , \\
\label{eqn:PatlakCPET}
\end{equation}

where the Patlak slope $\theta_1$ and the Patlak intercept $\theta_2$ are the model parameters that can be estimated from TAC measurements of ${C_{PET}}(t)$. 

%It is important to note a limitation of the Patlak model, which  that the estimated $K_i$ from the Patlak slope $\theta_1$ is susceptible to systematic errors in its estimation and can deviate from the true underlying $K_i (= \frac{K_1 k_3}{k_2+k_3})$.
A limitation in the use of the Patlak model using equation~\ref{eqn:PatlakCPET} is that $V_B$ is not necessarily known a priori and Patlak analysis can not distinguish between $K_i$ and $(1-V_B)$. In many applications $V_B$ is assumed to be small ($\leq$0.05) and neglected, but it can be a cause for systematic errors. 
In the applications of the Patlak model in this project, we will refer to the slope $\theta_1$ as the Patlak $K_i$ value, which is the value of interest that is commonly used in practice with Patlak analysis.


\section{Spectral analysis method - Basis pursuit}

The Spectral analysis method was originally introduced by Cunningham et al. \cite{Cunningham1993}. The method is based on the general form of the solutions of compartmental models that describe the impulse response function as a sum of positively-weighted (decaying) exponential functions, which are convolved with an input function to described the tissue activity behaviour.

The advantage of this methodology is that it can be used to fit on \gls{tac} data, with no prior knowledge and assumptions on the underlying kinetics. Furthermore, it can be used to deduce information of the underlying kinetics using the basis pursuit strategy proposed by Gunn \textit{et al.}~\cite{Gunn2002}.

Even without knowledge of the underlying kinetics


\begin{equation} 
C_{PET} = (1-V_B) C_T + V_B C_B
\end{equation}


where the $C_T$ can now be modelled with exponentials 

\begin{equation} 
C_{T} = IRF(t) \otimes C_A 
\end{equation}

\begin{equation} \label{IRF}
IRF(t) = \sum_{j=0}^{M} \phi_j e^{-\beta_j t}\
\end{equation}

The estimated spectral components assume different meaning depending on the position of the $\beta$ grid where they are located. When $\beta_j$ is very high ( $\lim_{j\to\infty} \beta_j$ ) the exponential is approximated by a delta function and represents free passage of tracer. When $\beta_j$ is very low ( $\beta_j=0$ ) the exponential is equal to 1 and represents full trapping of the tracer. 

We will use $\beta_{M+1} = \lim_{j\to\infty} \beta_j$ and $\beta_0 = 0 $. 
The tissue response function becomes:

\begin{equation}
IRF(t) = \phi_0 + \sum_{j=1}^{M} \phi_j e^{-\beta_j t} 
\end{equation}

and thus $C_T$ can be expressed as 

\begin{equation} 
C_{T}  = \phi_0  \int_{0}^{t}C_A  d\tau + \sum_{j=1}^{M} \phi_j e^{-\beta_j t} \otimes C_A 
\end{equation}

, where $\int_{0}^{t}C_A  d\tau$ represents full trapping of tracer.

For the PET measurement we have: 

\begin{equation} 
C_{PET} = ( \phi_0  \int_{0}^{t}C_A  d\tau + \sum_{j=1}^{M} \phi_j e^{-\beta_j t} \otimes C_A  )(1-V_B) + V_B C_B
\end{equation}


if we assume  $(1-V_B)\approx1$ then:

\begin{equation} 
C_{PET} = \phi_0\int_{0}^{t}C_Ad\tau  + \sum_{j=1}^{M} \phi_j e^{-\beta_j t} \otimes C_A  + V_B C_B
\end{equation}

Now assuming that the tracer is not metabolised and $C_A = C_B$, and re-write $V_B$ as $\phi_{M+1}$

\begin{equation} 
C_{PET} = \phi_0\int_{0}^{t}C_Ad\tau  + \sum_{j=1}^{M} \phi_j e^{-\beta_j t} \otimes C_A  + \phi_{M+1} C_A
\end{equation}

where $\phi_0$ represents trapping of the tracer, $\phi_j$ the M spectral coefficients that describe the equilibrating components (equal to the number of compartments in the system) and $\phi_{M+1}$ the blood fraction

For representation purposes this equation can be re-written in a generic form 

\begin{equation} 
C_{PET} = \sum_{j=0}^{M+1} \phi_j e^{-\beta_j t} \otimes C_A 
\end{equation}
