\section*{Résumé}
La tomographie par émission de positons (TEP) est fréquemment utilisée pour des applications cliniques, avec une majorité des pratiques reposant sur des mesures qualitatives et semi-quantitatives. Mais l'imagerie TEP a la capacité de fournir des informations fonctionnelles entièrement quantitatives sur les processus sous-jacents explorés, grâce à l'imagerie dynamique et à la modélisation cinétique. Ces informations quantitatives peuvent être utilisées comme biomarqueurs pour des applications cliniques, en particulier pour la médecine de précision. Des protocoles avec des positions du lit multiples, dédiés à l'imagerie dynamique du corps entier (DWB), ont été développés afin étendre le champ de vue effectif, au prix de restrictions dans le nombre d'acquisitions et la fréquence d'échantillonnage. L'objectif de cette thèse est d'améliorer la qualité de l'imagerie paramétrique du corps entier pour les applications d'imagerie DWB sur un système hybride TEP-IRM.

Dans notre première contribution, nous avons présenté le développement d'un protocole entièrement automatisé pour l'imagerie DWB sur un système TEP-IRM clinique, qui a permis de réduire les délais d'acquisition, ce qui se traduit par une augmentation du nombre d'acquisitions et de la fréquence d'échantillonnage. Le recours à l'automatisation complète a permis d'optimiser la planification des positions individuelles des lits, en utilisant au mieux le champ de vue effectif. 

Pour la deuxième contribution, nous avons développé des algorithmes de reconstruction dynamique dans un logiciel ouvert, et évalué les avantages offerts par l'utilisation de divers modèles cinétiques dans la reconstruction de données TEP dynamiques simulées et réelles. 

Nos résultats sont en accord avec les conclusions d’études antérieures sur l'utilisation de la reconstruction dynamique. Dans notre cas particulier de l'imagerie DWB, la reconstruction dynamique a montré des propriétés favorables pour l'exactitude et la précision des images paramétriques du corps entier, tout en fournissant des images dont le bruit est comparable à celui des protocoles dynamiques standards à position de lit unique, reconstruits avec des techniques ordinaires.

Dans notre troisième contribution, nous présentons une extension des fonctionnalités développées précédemment: la reconstruction dynamique simultanée de toutes les données multi-lits. Cette méthodologie permet l'utilisation synchrone de toutes les données d'acquisition DWB dans une seule boucle de reconstruction. La méthode a été appliquée à une étude pharmacocinétique DWB réalisée sur un système TEP-IRM clinique. Une comparaison a été faite avec des reconstructions statiques standards suivies d'une modélisation cinétique post reconstruction. Les résultats obtenus avec les deux méthodes étaient en bon accord, sans introduction de biais sur les métriques évaluées. En outre, l'utilisation de la reconstruction dynamique a entraîné une réduction notable du bruit dans les images d’émission et paramétriques. En outre, une méthode de détection et de correction des erreurs de modélisation utilisant la modélisation résiduelle adaptative a été appliquée et évaluée. Elle a montré des résultats prometteurs pour la réduction des erreurs de modélisation et leur propagation, tout en permettant la généricité dans l'utilisation des algorithmes de reconstruction dynamique.

Nos résultats ont montré que la reconstruction dynamique est nécessaire en imagerie paramétrique corps-entier pour obtenir une quantification précise et stable. De nombreuses méthodes ont été proposées dans ce projet afin d’optimiser le processus de reconstruction TEP pour l'imagerie DWB, en utilisant au mieux les données dynamiques acquises sur plusieurs positions de lit. Pour généraliser son utilisation, certaines améliorations méthodologiques doivent encore être apportées pour garantir une imagerie paramétrique fiable et sans artefact, notamment en ce qui concerne les mouvements du patient.
