\section*{Résumé}
La tomographie par émission de positons (TEP) est largement utilisée pour des applications cliniques, mais la majorité des pratiques se fonde sur des mesures qualitatives et semi-quantitatives. Mais la technique d'imagerie TEP permet de fournir des informations fonctionnelles complètement quantitatives sur les processus observés, grâce à l'imagerie dynamique et à la modélisation. Ces informations uniques peuvent être utilisées comme biomarqueurs pour des applications cliniques, en particulier pour la médecine de précision. Mais les applications cliniques de la TEP nécessitent souvent l'imagerie du corps entier et la majorité des scanners cliniques n'offrent qu'un champ de vue axial (CDV) limité. Des protocoles de positions multiples du lit pour l'imagerie dynamique du corps entier (DCE) ont été développés pour étendre le CDV effectif, au prix de limitations considérables du nombre d'acquisitions et de la fréquence d'échantillonnage. L'objectif de cette thèse est d'améliorer l'imagerie paramétrique du corps entier pour les applications d'imagerie DCE sur un scanner hybride TEP/IRM.

Dans notre première contribution, nous avons présenté le développement d'un protocole automatisé pour l'imagerie DWB sur le système clinique PET/MR, qui a permis de réduire les délais d'acquisition, ce qui se traduit en une augmentation du nombre d'acquisitions et de la fréquence d'échantillonnage. De plus, l'utilisation de l'automatisation complète a permis d'optimiser la planification de la position des lits individuels en utilisant au mieux le CDV effectif. 

Pour la deuxième contribution, nous avons développé des algorithmes de reconstruction dynamique dans un logiciel de reconstruction open source existant, et évalué les avantages offerts par l'utilisation de modèles dynamiques dans la reconstruction sur des données TEP simulées et réelles. 
La simulation et les données réelles étaient concentrées sur les applications oncologiques de la TEP. Les résultats ont confirmé les conclusions précédentes sur l'utilisation de la reconstruction dynamique. Dans le cas particulier de l'imagerie DCE, la reconstruction dynamique a montré des propriétés favorables à la précision et à l'exactitude des images paramétriques du corps entier, tout en fournissant des images dont le bruit est comparable à celui des protocoles dynamiques réguliers à lit unique traités avec les techniques de reconstruction habituelles.

Dans notre troisième contribution, nous présentons une extension des fonctionnalités développées sur le logiciel de reconstruction pour permettre une reconstruction dynamique multi-lits directe des données DWB. Cette méthodologie permet d'utiliser toutes les données d'une acquisition DCE dans une seule boucle de reconstruction. L'application de cette méthodologie à une étude pharmacocinétique réelle, la première chez l'homme, et la comparaison avec des reconstructions statiques suivies d'une modélisation paramétrique post-reconstruction ont montré un bon accord sans introduction de biais sur les paramètres évalués. En outre, l'utilisation de la reconstruction dynamique a permis de réduire sensiblement le bruit de l'activité et des images paramétriques. 
Dans cette application, une méthode de correction des erreurs de modélisation utilisant la modélisation résiduelle adaptative est également appliquée et évaluée, ce qui a montré des résultats prometteurs dans la réduction des erreurs de modélisation et de la propagation des erreurs tout en permettant la généricité dans l'utilisation des algorithmes de reconstruction dynamique.
