%%%%%%%%%%%%%%%%%%%%%%%%%%%%%%%%%%%%%%%%%%%%%%%%%%%%%%%%%%%%%%%%%%%%%%%%%%%%%%%%%%%%%%%%%%%%%%%%%%%%%%%%%%%%%%%%%%%%%%%%%%%%%%%%%%%%%%%%%%%%%%%%%%%%%%%%%%%%%%%%%%%%%%%
%%%%%%%%%%%%%%%%%%%%%%%%%%%%%%%%%%%%%%%%%%%%%%%%%%%%%%%%%%%%%%%%%%%%%%%%%%%%%%%%%%%%%%%%%%%%%%%%%%%%%%%%%%%%%%%%%%%%%%%%%%%%%%%%%%%%%%%%%%%%%%%%%%%%%%%%%%%%%%%%%%%%%%%
%%% Modèle pour la 4ème de couverture des thèses préparées à l'Université Paris-Saclay, basé sur le modèle produit par Nikolas STOTT / Template for back cover of thesis made at Université Paris-Saclay, based on the template made by Nikolas STOTT
%%% Mis à jour par Aurélien ARNOUX (École polytechnique)/ Updated by Aurélien ARNOUX (École polytechnique)
%%% Les instructions concernant chaque donnée à remplir sont données en bloc de commentaire / Rules to fill this file are given in comment blocks
%%% ATTENTION Ces informations doivent tenir sur une seule page une fois compilées / WARNING These informations must contain in no more than one page once compiled
%%%%%%%%%%%%%%%%%%%%%%%%%%%%%%%%%%%%%%%%%%%%%%%%%%%%%%%%%%%%%%%%%%%%%%%%%%%%%%%%%%%%%%%%%%%%%%%%%%%%%%%%%%%%%%%%%%%%%%%%%%%%%%%%%%%%%%%%%%%%%%%%%%%%%%%%%%%%%%%%%%%%%%%
%%% Version du 23 mai 2019 (Merci à Thibault CHEVALÉRIAS (CEA) pour ses suggestions et corrections)
%%%%%%%%%%%%%%%%%%%%%%%%%%%%%%%%%%%%%%%%%%%%%%%%%%%%%%%%%%%%%%%%%%%%%%%%%%%%%%%%%%%%%%%%%%%%%%%%%%%%%%%%%%%%%%%%%%%%%%%%%%%%%%%%%%%%%%%%%%%%%%%%%%%%%%%%%%%%%%%%%%%%%%%


\label{form_last}
%%%%%%%%%%%%%%%%%%%%%%%%%%%%%%%%%%%%%%%%%%%%%%%%%%%%%%%%%%%%%%%%%%%%%%%%%%%%%%%%%%%%%%%%%%%%%%%%%%%%%%%%%%%%%%%%%%%%%%%%%%%%%%%%%%%%%%%%%%%%%%%%%%%%%%%%%%%%%%%%%%%%%%%
%%%%%%%%%%%%%%%%%%%%%%%%%%%%%%%%%%%%%%%%%%%%%%%%%%%%%%%%%%%%%%%%%%%%%%%%%%%%%%%%%%%%%%%%%%%%%%%%%%%%%%%%%%%%%%%%%%%%%%%%%%%%%%%%%%%%%%%%%%%%%%%%%%%%%%%%%%%%%%%%%%%%%%%
%%% Formulaire / Form
%%% Remplacer les paramètres des \newcommand par les informations demandées / Replace \newcommand parameters by asked informations
%%%%%%%%%%%%%%%%%%%%%%%%%%%%%%%%%%%%%%%%%%%%%%%%%%%%%%%%%%%%%%%%%%%%%%%%%%%%%%%%%%%%%%%%%%%%%%%%%%%%%%%%%%%%%%%%%%%%%%%%%%%%%%%%%%%%%%%%%%%%%%%%%%%%%%%%%%%%%%%%%%%%%%%
%%%%%%%%%%%%%%%%%%%%%%%%%%%%%%%%%%%%%%%%%%%%%%%%%%%%%%%%%%%%%%%%%%%%%%%%%%%%%%%%%%%%%%%%%%%%%%%%%%%%%%%%%%%%%%%%%%%%%%%%%%%%%%%%%%%%%%%%%%%%%%%%%%%%%%%%%%%%%%%%%%%%%%%

\newcommand{\logoEd}{ed}																		%% Logo de l'école doctorale. Indiquer le sigle / Doctoral school logo. Indicate the acronym : 2MIB; AAIF; ABIES; BIOSIGNE; CBMS; EDMH; EDOM; EDPIF; EDSP; EOBE; INTERFACES; ITFA; PHENIICS; SDSV; SDV; SHS; SMEMAG; SSMMH; STIC
\newcommand{\PhDTitleFR}{Modélisation et reconstruction directe d'une image in vivo 3-D d'un paramètre pharmacologique en imagerie médicale hybride TEP et IRM}												%% Titre de la thèse en français / Thesis title in french
\newcommand{\keywordsFR}{3 à 6 mots clés}														%% Mots clés en français, séprarés par des , / Keywords in french, separated by ,
\newcommand{\abstractFR}{La tomographie par émission de positons (TEP) est largement utilisée pour des applications cliniques, mais la majorité des pratiques se fonde sur des mesures qualitatives et semi-quantitatives. Mais la technique d'imagerie TEP permet de fournir des informations fonctionnelles complètement quantitatives sur les processus observés, grâce à l'imagerie dynamique et à la modélisation. Ces informations uniques peuvent être utilisées comme biomarqueurs pour des applications cliniques, en particulier pour la médecine de précision. Mais les applications cliniques de la TEP nécessitent souvent l'imagerie du corps entier et la majorité des scanners cliniques n'offrent qu'un champ de vue axial (CDV) limité. Des protocoles de positions multiples du lit pour l'imagerie dynamique du corps entier (DCE) ont été développés pour étendre le CDV effectif, au prix de limitations considérables du nombre d'acquisitions et de la fréquence d'échantillonnage. L'objectif de cette thèse est d'améliorer l'imagerie paramétrique du corps entier pour les applications d'imagerie DCE sur un scanner hybride TEP/IRM.

Dans notre première contribution, nous avons présenté le développement d'un protocole automatisé pour l'imagerie DWB sur le système clinique PET/MR, qui a permis de réduire les délais d'acquisition, ce qui se traduit en une augmentation du nombre d'acquisitions et de la fréquence d'échantillonnage. De plus, l'utilisation de l'automatisation complète a permis d'optimiser la planification de la position des lits individuels en utilisant au mieux le CDV effectif. 

Pour la deuxième contribution, nous avons développé des algorithmes de reconstruction dynamique dans un logiciel de reconstruction open source existant, et évalué les avantages offerts par l'utilisation de modèles dynamiques dans la reconstruction sur des données TEP simulées et réelles. 
La simulation et les données réelles étaient concentrées sur les applications oncologiques de la TEP. Les résultats ont confirmé les conclusions précédentes sur l'utilisation de la reconstruction dynamique. Dans le cas particulier de l'imagerie DCE, la reconstruction dynamique a montré des propriétés favorables à la précision et à l'exactitude des images paramétriques du corps entier, tout en fournissant des images dont le bruit est comparable à celui des protocoles dynamiques réguliers à lit unique traités avec les techniques de reconstruction habituelles.

Dans notre troisième contribution, nous présentons une extension des fonctionnalités développées sur le logiciel de reconstruction pour permettre une reconstruction dynamique multi-lits directe des données DWB. Cette méthodologie permet d'utiliser toutes les données d'une acquisition DCE dans une seule boucle de reconstruction. L'application de cette méthodologie à une étude pharmacocinétique réelle, la première chez l'homme, et la comparaison avec des reconstructions statiques suivies d'une modélisation paramétrique post-reconstruction ont montré un bon accord sans introduction de biais sur les paramètres évalués. En outre, l'utilisation de la reconstruction dynamique a permis de réduire sensiblement le bruit de l'activité et des images paramétriques. 
Dans cette application, une méthode de correction des erreurs de modélisation utilisant la modélisation résiduelle adaptative est également appliquée et évaluée, ce qui a montré des résultats prometteurs dans la réduction des erreurs de modélisation et de la propagation des erreurs tout en permettant la généricité dans l'utilisation des algorithmes de reconstruction dynamique.}															%% Résumé en français / abstract in french

\newcommand{\PhDTitleEN}{Modelling and Reconstruction of a 3-D Whole Body parametric map in Hybrid PET-MRI Pharmacological imaging}													%% Titre de la thèse en anglais / Thesis title in english
\newcommand{\keywordsEN}{3 à 6 mots clés}														%% Mots clés en anglais, séprarés par des , / Keywords in english, separated by ,
\newcommand{\abstractEN}{Positron Emission Tomography (PET) is used extensively for clinical applications, with the majority of practices relying on qualitative and semi-quantitative measures. But PET imaging has the ability to deliver fully quantitative functional information of underlying imaged processes, by use of dynamic imaging and modelling. That unique information can be utilised as biomarkers for clinical applications, with special focus on precision medicine. But clinical applications of PET often require imaging over the whole body and the majority of clinical scanner provide only a limited axial field-of-view (FOV). Multiple bed position protocols for dynamic whole-body (DWB) imaging have been developed to extend the effective FOV, at the cost of considerable limitations in acquisition counts and sampling frequency.   The objective of this thesis is to improve whole body parametric imaging for DWB imaging applications on a hybrid PET/MR scanner.

In our first contribution we presented the development of a fully automated protocol for DWB imaging on the clinical PET/MR system, which resulted in reduced delays during acquisition that translate to increased acquisition counts and sampling frequency. Furthermore, use of full automation enabled for optimized planning of individual bed positions making best use of the effective FOV. 

For the second contribution we developed dynamic reconstruction algorithms within an existing open source reconstruction software, and evaluated on benefits offered by use of dynamic models in reconstruction on simulated and real PET data. 
The simulation and real data were focused on oncological applications of PET. Results agreed with previous findings on the use of dynamic reconstruction. In the particular case of DWB imaging dynamic reconstruction showed desirable properties for whole-body parametric image accuracy and precision, while providing images of comparable image noise to regular single bed dynamic protocols processed with regular reconstruction techniques.

In our third contribution we present an extension of the developed functionalities on reconstruction software to enable direct multi-bed dynamic reconstruction of DWB data. This methodology enables the use of all data of an DWB acquisition to be used within a single reconstruction loop. Application of this methodology on a real, first in man, DWB pharmacokinetic study and comparison with regular frame static reconstructions followed by post reconstruction parametric modelling showed good agreement with no introduction of bias on the evaluated metrics. Furthermore, the use of dynamic reconstruction resulted in noticeable noise reduction of the activity and parametric images. 
In this application an modelling error correction method using adaptive residual modelling is also applied and evaluated, which showed promising results in reducing modelling errors and error propagation while allowing for genericity in the use of dynamic reconstruction algorithms.}															%% Résumé en anglais / abstract in english

\label{layout_last}
%%%%%%%%%%%%%%%%%%%%%%%%%%%%%%%%%%%%%%%%%%%%%%%%%%%%%%%%%%%%%%%%%%%%%%%%%%%%%%%%%%%%%%%%%%%%%%%%%%%%%%%%%%%%%%%%%%%%%%%%%%%%%%%%%%%%%%%%%%%%%%%%%%%%%%%%%%%%%%%%%%%%%%%
%%%%%%%%%%%%%%%%%%%%%%%%%%%%%%%%%%%%%%%%%%%%%%%%%%%%%%%%%%%%%%%%%%%%%%%%%%%%%%%%%%%%%%%%%%%%%%%%%%%%%%%%%%%%%%%%%%%%%%%%%%%%%%%%%%%%%%%%%%%%%%%%%%%%%%%%%%%%%%%%%%%%%%%
%%% Mise en page / Page layout      
%%% NE RIEN MODIFIER / DO NOT MODIFY
%%%%%%%%%%%%%%%%%%%%%%%%%%%%%%%%%%%%%%%%%%%%%%%%%%%%%%%%%%%%%%%%%%%%%%%%%%%%%%%%%%%%%%%%%%%%%%%%%%%%%%%%%%%%%%%%%%%%%%%%%%%%%%%%%%%%%%%%%%%%%%%%%%%%%%%%%%%%%%%%%%%%%%%
%%%%%%%%%%%%%%%%%%%%%%%%%%%%%%%%%%%%%%%%%%%%%%%%%%%%%%%%%%%%%%%%%%%%%%%%%%%%%%%%%%%%%%%%%%%%%%%%%%%%%%%%%%%%%%%%%%%%%%%%%%%%%%%%%%%%%%%%%%%%%%%%%%%%%%%%%%%%%%%%%%%%%%%

\pagestyle{empty}

%%% Logo de l'école doctorale. Le nom du fichier correspond au sigle de l'ED / Doctoral school logo. Filename correspond to doctoral school acronym
%%% Les noms valides sont / Valid names are : 2MIB; AAIF; ABIES; BIOSIGNE; CBMS; EDMH; EDOM; EDPIF; EDSP; EOBE; INTERFACES; ITFA; PHENIICS; SDSV; SDV; SHS; SMEMAG; SSMMH; STIC
\begin{textblock*}{61mm}(16mm,3mm)
    \textblockcolour{white}
	\noindent\includegraphics[height=24mm]{media/ed/\logoEd.jpeg}
\end{textblock*}



%%%Titre de la thèse en français / Thesis title in french
\begin{singlespace}
\begin{center}
\fcolorbox{bordeau}{white}{\parbox{0.95\textwidth}{
{\bf Titre:} \PhDTitleFR 
\medskip

%%%Mots clés en français, séprarés par des ; / Keywords in french, separated by ;
{\bf Mots clés:} \keywordsFR 
\vspace{-2mm}

%%% Résumé en français / abstract in french
\begin{multicols}{2}
{\bf Résumé:} 
\abstractFR 
\end{multicols}
}}
\end{center}

\vspace*{0mm}

%%%Titre de la thèse en anglais / Thesis title in english
\begin{center}
\fcolorbox{bordeau}{white}{\parbox{0.95\textwidth}{
{\bf Title:} \PhDTitleEN 

\medskip

%%%Mots clés en anglais, séprarés par des ; / Keywords in english, separated by ;
{\bf Keywords:}  \keywordsEN %%3 à 6 mots clés%%
\vspace{-2mm}
\begin{multicols}{2}
	
%%% Résumé en anglais / abstract in english
{\bf Abstract:} 
\abstractEN
\end{multicols}
}}
\end{center}

\begin{textblock*}{161mm}(10mm,270mm)
\textblockcolour{white}
\color{bordeau}
{\bf\noindent Université Paris-Saclay	         }

\noindent Espace Technologique / Immeuble Discovery 

\noindent Route de l’Orme aux Merisiers RD 128 / 91190 Saint-Aubin, France 
\end{textblock*}

\begin{textblock*}{0mm}(182mm,255mm)
\textblockcolour{white}
\includegraphics[width=20mm]{media/UPSACLAY-petit}
\end{textblock*}
\end{singlespace}