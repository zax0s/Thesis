%%%%%%%%%%%%%%%%%%%%%%%%%%%%%%%%%%%%%%%%%%%%%%%%%%%%%%%%%%%%%%%%%%%%%%%%%%%%%%%%%%%%%%%%%%%%%%%%%%%%%%%%%%%%%%%%%%%%%%%%%%%%%%%%%%%%%%%%%%%%%%%%%%%%%%%%%%%%%%%%%%%%%%%
%%%%%%%%%%%%%%%%%%%%%%%%%%%%%%%%%%%%%%%%%%%%%%%%%%%%%%%%%%%%%%%%%%%%%%%%%%%%%%%%%%%%%%%%%%%%%%%%%%%%%%%%%%%%%%%%%%%%%%%%%%%%%%%%%%%%%%%%%%%%%%%%%%%%%%%%%%%%%%%%%%%%%%%
%%% Modèle pour la 4ème de couverture des thèses préparées à l'Université Paris-Saclay, basé sur le modèle produit par Nikolas STOTT / Template for back cover of thesis made at Université Paris-Saclay, based on the template made by Nikolas STOTT
%%% Mis à jour par Aurélien ARNOUX (École polytechnique)/ Updated by Aurélien ARNOUX (École polytechnique)
%%% Les instructions concernant chaque donnée à remplir sont données en bloc de commentaire / Rules to fill this file are given in comment blocks
%%% ATTENTION Ces informations doivent tenir sur une seule page une fois compilées / WARNING These informations must contain in no more than one page once compiled
%%%%%%%%%%%%%%%%%%%%%%%%%%%%%%%%%%%%%%%%%%%%%%%%%%%%%%%%%%%%%%%%%%%%%%%%%%%%%%%%%%%%%%%%%%%%%%%%%%%%%%%%%%%%%%%%%%%%%%%%%%%%%%%%%%%%%%%%%%%%%%%%%%%%%%%%%%%%%%%%%%%%%%%
%%% Version du 23 mai 2019 (Merci à Thibault CHEVALÉRIAS (CEA) pour ses suggestions et corrections)
%%%%%%%%%%%%%%%%%%%%%%%%%%%%%%%%%%%%%%%%%%%%%%%%%%%%%%%%%%%%%%%%%%%%%%%%%%%%%%%%%%%%%%%%%%%%%%%%%%%%%%%%%%%%%%%%%%%%%%%%%%%%%%%%%%%%%%%%%%%%%%%%%%%%%%%%%%%%%%%%%%%%%%%


\label{form_last}
%%%%%%%%%%%%%%%%%%%%%%%%%%%%%%%%%%%%%%%%%%%%%%%%%%%%%%%%%%%%%%%%%%%%%%%%%%%%%%%%%%%%%%%%%%%%%%%%%%%%%%%%%%%%%%%%%%%%%%%%%%%%%%%%%%%%%%%%%%%%%%%%%%%%%%%%%%%%%%%%%%%%%%%
%%%%%%%%%%%%%%%%%%%%%%%%%%%%%%%%%%%%%%%%%%%%%%%%%%%%%%%%%%%%%%%%%%%%%%%%%%%%%%%%%%%%%%%%%%%%%%%%%%%%%%%%%%%%%%%%%%%%%%%%%%%%%%%%%%%%%%%%%%%%%%%%%%%%%%%%%%%%%%%%%%%%%%%
%%% Formulaire / Form
%%% Remplacer les paramètres des \newcommand par les informations demandées / Replace \newcommand parameters by asked informations
%%%%%%%%%%%%%%%%%%%%%%%%%%%%%%%%%%%%%%%%%%%%%%%%%%%%%%%%%%%%%%%%%%%%%%%%%%%%%%%%%%%%%%%%%%%%%%%%%%%%%%%%%%%%%%%%%%%%%%%%%%%%%%%%%%%%%%%%%%%%%%%%%%%%%%%%%%%%%%%%%%%%%%%
%%%%%%%%%%%%%%%%%%%%%%%%%%%%%%%%%%%%%%%%%%%%%%%%%%%%%%%%%%%%%%%%%%%%%%%%%%%%%%%%%%%%%%%%%%%%%%%%%%%%%%%%%%%%%%%%%%%%%%%%%%%%%%%%%%%%%%%%%%%%%%%%%%%%%%%%%%%%%%%%%%%%%%%

\newcommand{\logoEd}{ed}																		%% Logo de l'école doctorale. Indiquer le sigle / Doctoral school logo. Indicate the acronym : 2MIB; AAIF; ABIES; BIOSIGNE; CBMS; EDMH; EDOM; EDPIF; EDSP; EOBE; INTERFACES; ITFA; PHENIICS; SDSV; SDV; SHS; SMEMAG; SSMMH; STIC
\newcommand{\PhDTitleFR}{Modélisation et reconstruction d'images paramétriques du corps entier en imagerie pharmacologique TEP-IRM}	%% Titre de la thèse en français / Thesis title in french
\newcommand{\keywordsFR}{TEP, reconstruction dynamique, TEP corps entier, imagerie paramétrique} %% Mots clés en français, séprarés par des , / Keywords in french, separated by ,
\newcommand{\abstractFR}{La tomographie par émission de positons (TEP) est fréquemment utilisée pour des applications cliniques, avec une majorité des pratiques reposant sur des mesures qualitatives et semi-quantitatives. Mais l'imagerie TEP a la capacité de fournir des informations fonctionnelles entièrement quantitatives sur les processus sous-jacents explorés, grâce à l'imagerie dynamique et à la modélisation cinétique. Ces informations quantitatives peuvent être utilisées comme biomarqueurs pour des applications cliniques, en particulier pour la médecine de précision. Des protocoles avec des positions du lit multiples, dédiés à l'imagerie dynamique du corps entier (DWB), ont été développés afin d'étendre le champ de vue effectif, au prix de restrictions dans le nombre de détections et la fréquence d'échantillonnage. L'objectif de cette thèse est d'améliorer la qualité de l'imagerie paramétrique du corps entier pour les applications d'imagerie DWB sur un système hybride TEP-IRM.

Dans notre première contribution, nous avons présenté le développement d'un protocole entièrement automatisé pour l'imagerie DWB sur un système TEP-IRM clinique, qui a permis de réduire les délais d'acquisition, ce qui se traduit par une augmentation du nombre de détections et de la fréquence d'échantillonnage. Le recours à l'automatisation complète a permis d'optimiser la planification des positions des lits, en utilisant au mieux le champ de vue effectif. 

Pour la deuxième contribution, nous avons développé des algorithmes de reconstruction dynamique dans un logiciel ouvert, et évalué les avantages offerts par l'utilisation de divers modèles cinétiques dans la reconstruction de données TEP dynamiques simulées et réelles. Ces évaluations étaient focalisées sur les reconstructions de lits individuels des protocoles DWB. Dans le cas particulier de l'imagerie DWB, la reconstruction dynamique a montré des propriétés favorables pour l'exactitude et la précision des images paramétriques du corps entier, tout en fournissant des images dont le bruit est comparable à celui des protocoles dynamiques standards à position de lit unique, reconstruits avec des techniques ordinaires.

Dans notre troisième contribution, nous présentons une extension des fonctionnalités développées précédemment: la reconstruction dynamique simultanée de toutes les données multi-lits. Cette méthodologie permet l'utilisation synchrone de toutes les données d'acquisition DWB dans une seule boucle de reconstruction. La méthode a été appliquée à une étude pharmacocinétique DWB réalisée sur un système TEP-IRM. Une comparaison a été faite avec des reconstructions statiques standards suivies d'une modélisation cinétique post reconstruction. Les résultats obtenus avec les deux méthodes étaient en bon accord, sans introduction de biais sur les métriques évaluées. En outre, l'utilisation de la reconstruction dynamique a entraîné une réduction notable du bruit dans les images d’émission et paramétriques. 

Dans notre quatrième contribution, une méthode de détection et de correction des erreurs de modélisation utilisant la modélisation résiduelle adaptative a été appliquée et évaluée. Elle a montré des résultats prometteurs pour la réduction des erreurs de modélisation et leur propagation, tout en permettant la généricité dans l'utilisation des algorithmes de reconstruction dynamique.

Nos résultats ont montré que la reconstruction dynamique est nécessaire en imagerie paramétrique corps-entier pour obtenir une quantification précise et stable. De nombreuses méthodes ont été proposées dans ce projet afin d’optimiser le processus de reconstruction TEP pour l'imagerie DWB, en utilisant au mieux les données dynamiques acquises sur plusieurs positions de lit. Pour généraliser son utilisation, certaines améliorations méthodologiques doivent encore être apportées pour garantir une imagerie paramétrique fiable et sans artefact, notamment en ce qui concerne les mouvements du patient.}															%% Résumé en français / abstract in french

\newcommand{\PhDTitleEN}{Modelling and Reconstruction of \mbox{Whole Body} parametric maps in PET-MRI Pharmacological imaging}													%% Titre de la thèse en anglais / Thesis title in english
\newcommand{\keywordsEN}{PET, dynamic reconstruction, whole body PET, parametric imaging}														%% Mots clés en anglais, séprarés par des , / Keywords in english, separated by ,
\newcommand{\abstractEN}{Positron Emission Tomography (PET) is used extensively for clinical applications, with the majority of practices relying on qualitative and semi-quantitative measures. But PET imaging has the ability to deliver fully quantitative functional information of underlying imaged processes by use of dynamic imaging and kinetic modelling. That unique quantitative information can be utilised as biomarkers for clinical applications, especially in precision medicine. But clinical applications often require imaging over the whole body and the majority of clinical scanners provide a limited axial Field of View (FOV). Multiple bed position protocols for dynamic whole-body (DWB) imaging have been developed to extend the effective FOV, at the cost of considerable limitations in acquisition counts and sampling frequency. The objective of this thesis is to improve the quality of whole-body parametric imaging for DWB imaging applications on a hybrid PET/MR scanner.

In our first contribution we presented the development of a fully automated acquisition protocol for DWB imaging on a clinical PET/MR system, which resulted in reduced delays between whole body sweeps of the dynamic whole-body acquisition. These improvements can be used towards increasing acquisition counts and sampling frequency. Furthermore, the use of full automation enabled optimized planning of the individual bed positions for the best use of the effective FOV. 

For the second contribution we developed dynamic reconstruction algorithms within an existing open source reconstruction software. We evaluated benefits offered by use of various dynamic models in reconstruction, using simulated and real dynamic PET data. These evaluations were focused on reconstructions of individual beds from DWB acquisition protocols.
Our results agreed with previous findings on the use of dynamic reconstruction. In the particular case of DWB imaging, dynamic reconstruction showed desirable properties for whole-body parametric image accuracy and precision, while providing images of comparable image noise to regular single bed dynamic protocols processed with regular reconstruction techniques.

In our third contribution we present an extension of the developed functionalities on the reconstruction software for direct multi-bed dynamic reconstruction of DWB data. This methodology enables the synchronous use of all acquired DWB data within a single reconstruction loop. The method was applied on a DWB pharmacological study performed on a clinical PET/MR system and comparison was made with regular frame static reconstructions followed by post reconstruction parametric modelling. The results between the two methods were in good agreement, with no introduction of bias on the evaluated metrics. Furthermore, the use of dynamic reconstruction resulted in noticeable noise reduction in activity and parametric images.

In the fourth and final contribution, an adaptive residual modelling method was applied in reconstruction and evaluated on the DWB pharmacological study, to address modelling errors. This method showed promising results in reducing modelling errors and error propagation while also allowing for genericity in the use of dynamic reconstruction algorithms.

Overall, our findings showed that dynamic reconstruction is necessary in DWB parametric imaging to achieve accurate and stable quantification. Many methods have been proposed in this project that showed how reconstruction can be optimised for multi-bed DWB imaging, by making best use of all dynamic and bed PET raw data in the reconstruction process. But before widespread use of dynamic reconstruction, some methodological improvements need to be addressed further to guarantee artefact free and reliable parametric imaging. Most notably there is need for accurate estimation of underlying complex elastic motion in the dynamic datasets, followed by the correction of these motion types within the dynamic reconstruction process.}															%% Résumé en anglais / abstract in english

\label{layout_last}
%%%%%%%%%%%%%%%%%%%%%%%%%%%%%%%%%%%%%%%%%%%%%%%%%%%%%%%%%%%%%%%%%%%%%%%%%%%%%%%%%%%%%%%%%%%%%%%%%%%%%%%%%%%%%%%%%%%%%%%%%%%%%%%%%%%%%%%%%%%%%%%%%%%%%%%%%%%%%%%%%%%%%%%
%%%%%%%%%%%%%%%%%%%%%%%%%%%%%%%%%%%%%%%%%%%%%%%%%%%%%%%%%%%%%%%%%%%%%%%%%%%%%%%%%%%%%%%%%%%%%%%%%%%%%%%%%%%%%%%%%%%%%%%%%%%%%%%%%%%%%%%%%%%%%%%%%%%%%%%%%%%%%%%%%%%%%%%
%%% Mise en page / Page layout      
%%% NE RIEN MODIFIER / DO NOT MODIFY
%%%%%%%%%%%%%%%%%%%%%%%%%%%%%%%%%%%%%%%%%%%%%%%%%%%%%%%%%%%%%%%%%%%%%%%%%%%%%%%%%%%%%%%%%%%%%%%%%%%%%%%%%%%%%%%%%%%%%%%%%%%%%%%%%%%%%%%%%%%%%%%%%%%%%%%%%%%%%%%%%%%%%%%
%%%%%%%%%%%%%%%%%%%%%%%%%%%%%%%%%%%%%%%%%%%%%%%%%%%%%%%%%%%%%%%%%%%%%%%%%%%%%%%%%%%%%%%%%%%%%%%%%%%%%%%%%%%%%%%%%%%%%%%%%%%%%%%%%%%%%%%%%%%%%%%%%%%%%%%%%%%%%%%%%%%%%%%

\pagestyle{empty}

%%% Logo de l'école doctorale. Le nom du fichier correspond au sigle de l'ED / Doctoral school logo. Filename correspond to doctoral school acronym
%%% Les noms valides sont / Valid names are : 2MIB; AAIF; ABIES; BIOSIGNE; CBMS; EDMH; EDOM; EDPIF; EDSP; EOBE; INTERFACES; ITFA; PHENIICS; SDSV; SDV; SHS; SMEMAG; SSMMH; STIC
\begin{textblock*}{61mm}(16mm,3mm)
    \textblockcolour{white}
	\noindent\includegraphics[height=24mm]{media/ed/\logoEd.jpeg}
\end{textblock*}



%%%Titre de la thèse en français / Thesis title in french
\begin{singlespace}
\begin{center}
\fcolorbox{bordeau}{white}{\parbox{0.95\textwidth}{
{\bf Titre:} \PhDTitleFR 
\medskip

%%%Mots clés en français, séprarés par des ; / Keywords in french, separated by ;
{\bf Mots clés:} \keywordsFR 
\vspace{-2mm}

%%% Résumé en français / abstract in french
\begin{multicols}{2}
{\bf Résumé:} 
\abstractFR 
\end{multicols}
}}
\end{center}

\vspace*{0mm}

%%%Titre de la thèse en anglais / Thesis title in english
\begin{center}
\fcolorbox{bordeau}{white}{\parbox{0.95\textwidth}{
{\bf Title:} \PhDTitleEN 

\medskip

%%%Mots clés en anglais, séprarés par des ; / Keywords in english, separated by ;
{\bf Keywords:}  \keywordsEN %%3 à 6 mots clés%%
\vspace{-2mm}
\begin{multicols}{2}
	
%%% Résumé en anglais / abstract in english
{\bf Abstract:} 
\abstractEN
\end{multicols}
}}
\end{center}

\begin{textblock*}{161mm}(10mm,270mm)
\textblockcolour{white}
\color{bordeau}
{\bf\noindent Université Paris-Saclay	         }

\noindent Espace Technologique / Immeuble Discovery 

\noindent Route de l’Orme aux Merisiers RD 128 / 91190 Saint-Aubin, France 
\end{textblock*}

\begin{textblock*}{0mm}(182mm,255mm)
\textblockcolour{white}
\includegraphics[width=20mm]{media/UPSACLAY-petit}
\end{textblock*}
\end{singlespace}