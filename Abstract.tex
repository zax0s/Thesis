\section*{Abstract}
Positron Emission Tomography (PET) is used extensively for clinical applications, with the majority of practices relying on qualitative and semi-quantitative measures. But PET imaging has the ability to deliver fully quantitative functional information of underlying imaged processes by use of dynamic imaging and modelling. That unique quantitative information can be utilised as biomarkers for clinical applications, with special focus on precision medicine. Multiple bed position protocols for dynamic whole-body (DWB) imaging have been developed to extend the effective FOV, at the cost of considerable limitations in acquisition counts and sampling frequency. The objective of this thesis is to improve the quality of whole body parametric imaging for DWB imaging applications on a hybrid PET/MR scanner.

In our first contribution we presented the development of a fully automated protocol for DWB imaging on a clinical PET/MR system, which resulted in reduced delays during acquisition that translates to increased acquisition counts and sampling frequency. Use of full automation enabled optimized planning of individual bed positions, making best use of the effective FOV. 

For the second contribution we developed dynamic reconstruction algorithms within an existing open source reconstruction software. We evaluated on benefits offered by use of various dynamic models in reconstruction on simulated and real PET dynamic data using single-bed positions and considering DWB protocol timing.
Results agreed with previous findings on the use of dynamic reconstruction. In the particular case of DWB imaging dynamic reconstruction showed desirable properties for whole-body parametric image accuracy and precision, while providing images of comparable image noise to regular single bed dynamic protocols processed with regular reconstruction techniques.

In our third contribution we present an extension of the developed functionalities on the reconstruction software for direct multi-bed dynamic reconstruction of DWB data. This methodology enables the synchronous use of all DWB acquisition data within a single reconstruction loop. The method was applied on a DWB pharmacokinetic study performed on a clinical PET/MR system and comparison was made with regular frame static reconstructions followed by post reconstruction parametric modelling. The results between the two methods were in good agreement, with no introduction of bias on the evaluated metrics. Furthermore, the use of dynamic reconstruction resulted in noticeable noise reduction in the activity and parametric images. In this application a modelling error correction method using adaptive residual modelling was also applied and evaluated, which showed promising results in reducing modelling errors and error propagation while also allowing for genericity in the use of dynamic reconstruction algorithms.

Overall, our findings showed that dynamic reconstruction is necessary in DWB parametric imaging to achieve accurate and stable quantification. Many methods have been proposed in this project that showed how reconstruction can be optimised for multi-bed DWB imaging, by making best use of all dynamic and bed PET raw data in the reconstruction process. But before widespread use of dynamic reconstruction some methodological improvements need to be addressed further to guarantee artefact free and reliable parametric imaging. Most notably there is need for accurate estimation of underlying complex elastic motion in the dynamic datasets, followed by the correction of these motion types within the dynamic reconstruction process.
