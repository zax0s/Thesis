\section*{Abstract}

Positron Emission Tomography (PET) is used extensively for clinical applications, with the majority of practices relying on qualitative and semi-quantitative measures. But PET imaging has the ability to deliver fully quantitative functional information of underlying imaged processes, by use of dynamic imaging and modelling. That unique information can be utilised as biomarkers for clinical applications, with special focus on precision medicine. But clinical applications of PET often require imaging over the whole body and the majority of clinical scanner provide only a limited axial field-of-view (FOV). Multiple bed position protocols for dynamic whole-body (DWB) imaging have been developed to extend the effective FOV, at the cost of considerable limitations in acquisition counts and sampling frequency.   The objective of this thesis is to improve whole body parametric imaging for DWB imaging applications on a hybrid PET/MR scanner.

In our first contribution we presented the development of a fully automated protocol for DWB imaging on the clinical PET/MR system, which resulted in reduced delays during acquisition that translate to increased acquisition counts and sampling frequency. Furthermore, use of full automation enabled for optimized planning of individual bed positions making best use of the effective FOV. 

For the second contribution we developed dynamic reconstruction algorithms within an existing open source reconstruction software, and evaluated on benefits offered by use of dynamic models in reconstruction on simulated and real PET data. 
The simulation and real data were focused on oncological applications of PET. Results agreed with previous findings on the use of dynamic reconstruction. In the particular case of DWB imaging dynamic reconstruction showed desirable properties for whole-body parametric image accuracy and precision, while providing images of comparable image noise to regular single bed dynamic protocols processed with regular reconstruction techniques.

In our third contribution we present an extension of the developed functionalities on reconstruction software to enable direct multi-bed dynamic reconstruction of DWB data. This methodology enables the use of all data of an DWB acquisition to be used within a single reconstruction loop. Application of this methodology on a real, first in man, DWB pharmacokinetic study and comparison with regular frame static reconstructions followed by post reconstruction parametric modelling showed good agreement with no introduction of bias on the evaluated metrics. Furthermore, the use of dynamic reconstruction resulted in noticeable noise reduction of the activity and parametric images. 
In this application an modelling error correction method using adaptive residual modelling is also applied and evaluated, which showed promising results in reducing modelling errors and error propagation while allowing for genericity in the use of dynamic reconstruction algorithms.

